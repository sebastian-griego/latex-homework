\documentclass[12pt]{article}
\usepackage[utf8]{inputenc}
\usepackage{amsmath, amssymb, amsthm}
\usepackage{enumitem}
\usepackage{geometry}
\usepackage{fancyhdr}
\usepackage{pgfplots}
\usepackage{tikz}
\usepackage{float}
\usepackage{graphicx}
\DeclareMathOperator{\Tr}{Tr}
\DeclareMathOperator{\rng}{rng}
\DeclareMathOperator{\norm}{||}
\DeclareMathOperator{\NN}{\mathbb{N}}
\DeclareMathOperator{\ZZ}{\mathbb{Z}}
\DeclareMathOperator{\QQ}{\mathbb{Q}}
\DeclareMathOperator{\RR}{\mathbb{R}}
\DeclareMathOperator{\CC}{\mathbb{C}}

% Page setup
\setlength{\headheight}{15pt}
\geometry{letterpaper, margin=1in}
\setlength{\parindent}{0pt}
\setlength{\parskip}{1em}
\pagestyle{fancy}
\fancyhf{}
\fancyhead[L]{\textbf{Sebastian Griego}}  % Replace with your name
\fancyhead[C]{\textbf{Math Modeling}}  % Replace with your course name
\fancyhead[R]{\textbf{Assignment \#2}}  % Replace with your assignment number
\fancyfoot[C]{\thepage}

\newenvironment{problem}[1]{
    \textbf{Problem #1:}
}{
    \rmfamily \vspace{1em}
}

\newenvironment{solution}{
    \textbf{Solution:}
    
}{
    
    \vspace{2em}
}

\begin{document}

\title{Homework \#2}  % Replace with the homework number
\author{Sebastian Griego}  % Replace with your name

\begin{problem}{3}
    A hypothetical reaction in the study of isothermal autocatalytic reactions was considered by Gray and Scott (1985), whose kinetics in dimensionless form are given as follows:
    \[  
        \begin{aligned}
            \frac{dx}{dt} &= a(1-x)-xy^2\\
            \frac{dy}{dt} &= xy^2 - (a+k)y
        \end{aligned}
    \]
    where \(a\) and \(k\) are positive parameters. Show that the saddle node bifurcation occurs at \(k = -a \pm \frac{\sqrt{a}}{a}\).
\end{problem}

\begin{solution}
    Label the equations:
    \[
    \begin{aligned}
        \frac{dx}{dt} &= a(1 - x) - x y^2 \quad \text{(1)} \\
        \frac{dy}{dt} &= x y^2 - (a + k) y \quad \text{(2)}
    \end{aligned}
    \]
    
    Set both derivatives to zero to find the equilibrium points.
    
    From equation (2):
    \[
        \begin{aligned}
            x y^2 - (a + k) y &= 0\\
            y (x y - (a + k)) &= 0
        \end{aligned}
    \]
    This gives two cases:
    \begin{enumerate}
        \item \( y = 0 \) \\
        Substitute \( y = 0 \) into equation (1):
        \[
        a(1 - x) = 0 \\
        \Rightarrow x = 1
        \]
        So, one equilibrium point is \( (1, 0) \).
        
        \item \( x y = a + k \) \\
        Solve for \( x \):
        \[
        x = \frac{a + k}{y}
        \]
        Substitute \( x = \frac{a + k}{y} \) into equation (1):
        \[
        \begin{aligned}
            a\left(1 - \frac{a + k}{y}\right) - \left(\frac{a + k}{y}\right) y^2 &= 0 \\
            a - \frac{a(a + k)}{y} - (a + k) y &= 0 \\
            a y - a(a + k) - (a + k) y^2 &= 0 \\
            (a + k) y^2 - a y + a(a + k) &= 0
        \end{aligned}
        \]
    \end{enumerate}
    Set up the quadratic equation for \( y \):
    \[
        y = \frac{a \pm \sqrt{a^2 - 4a(a + k)^2}}{2(a + k)}
    \]

    A bifurcation occurs when the discriminant of the quadratic equation is zero.
    
    Set \( D = 0 \):
    \[
        \begin{aligned}
            a^2 - 4a(a + k)^2 &= 0 \\
            a - 4(a + k)^2 &= 0 \\
            4(a + k)^2 &= a \\
            (a + k)^2 &= \frac{a}{4} \\
            a + k &= \pm \frac{\sqrt{a}}{2} \\
            k &= -a \pm \frac{\sqrt{a}}{2}
        \end{aligned}
    \]
    The quadratic equation would change the sign of \(y\), so this is a saddle node bifurcation, and it occurs at \(k = -a \pm \frac{\sqrt{a}}{2}\).
\end{solution}


\end{document}

