\documentclass[12pt]{article}
\usepackage[utf8]{inputenc}
\usepackage{amsmath, amssymb, amsthm}
\usepackage{enumitem}
\usepackage{geometry}
\usepackage{fancyhdr}
\usepackage{pgfplots}
\usepackage{tikz}
\usepackage{float}
\usepackage{graphicx}
\usepackage{listings}
\DeclareMathOperator{\Tr}{Tr}
\DeclareMathOperator{\RR}{\mathbb{R}}
\DeclareMathOperator{\rng}{rng}

% Page setup
\setlength{\headheight}{15pt}
\geometry{letterpaper, margin=1in}
\setlength{\parindent}{0pt}
\setlength{\parskip}{1em}
\pagestyle{fancy}
\fancyhf{}
\fancyhead[L]{\textbf{Sebastian Griego}}  % Replace with your name
\fancyhead[C]{\textbf{Linear Algebra}}  % Replace with your course name
\fancyhead[R]{\textbf{Assignment \#1}}  % Replace with your assignment number
\fancyfoot[C]{\thepage}

\newenvironment{problem}[1]{
    \textbf{Problem #1:}
}{
    \rmfamily \vspace{1em}
}

\newenvironment{solution}{
    \textbf{Solution:}
    
}{
    
    \vspace{2em}
}

\begin{document}


\title{Homework \#1}  % Replace with the homework number
\author{Sebastian Griego}  % Replace with your name

\maketitle

\begin{problem}{1} Use \(A = UDV^T\) to recover the first column of \(A\). Show detailed calculations of all the relevant matrices and vectors, and use space-time decomposition to describe your results.
    For extra credit: Describe the spatial and temporal modes, and their corresponding variances or energies.
\end{problem}

\begin{solution}
    \[
        \begin{aligned}
            D &= \begin{pmatrix}
                1.414214 & 0 \\
                0 & 1.414214 \\
            \end{pmatrix} \quad \text{Diagonal Values} \\
            U &= \begin{pmatrix}
                -0.707107 & -0.707107 \\
                -0.707107 & 0.707107 \\
            \end{pmatrix} \quad \text{Spatial Pattern Matrix} \\
            V^T &= \begin{pmatrix}
                -1 & 0 \\
                0 & -1 \\
            \end{pmatrix} \quad \text{Temporal Pattern Matrix}
        \end{aligned}
    \]
    Calculate \(UDV^T\):
    \[
        \begin{aligned}
            UDV^T &= \begin{pmatrix}
                -0.707107 & -0.707107 \\
                -0.707107 & 0.707107 \\
            \end{pmatrix}
            \begin{pmatrix}
                1.414214 & 0 \\
                0 & 1.414214 \\
            \end{pmatrix}
            \begin{pmatrix}
                -1 & 0 \\
                0 & -1 \\
            \end{pmatrix} \\
            & = \begin{pmatrix}
                -0.707107 & -0.707107 \\
                -0.707107 & 0.707107 \\
            \end{pmatrix}
            \begin{pmatrix}
                1.414214(-1) + 0(0) & 1.414214(0) + 0(-1) \\
                0(-1) + 1.414214(0) & 0(0) + 1.414214(-1) \\
            \end{pmatrix} \\
            & = \begin{pmatrix}
                -0.707107 & -0.707107 \\
                -0.707107 & 0.707107 \\
            \end{pmatrix}
            \begin{pmatrix}
                -1.414214 & 0 \\
                0 & -1.414214 \\
            \end{pmatrix} \\
            & = \begin{pmatrix}
                -0.707107(-1.414214) + (-0.707107)(0) & -0.707107(0) + (-0.707107)(-1.414214) \\
                -0.707107(-1.414214) + 0.707107(0) & -0.707107(0) + 0.707107(-1.414214) \\
            \end{pmatrix} \\
            & = \begin{pmatrix}
                1 & 1 \\
                1 & -1 \\
            \end{pmatrix}
        \end{aligned}
    \]
    So, the first column of \(A\) is \(\begin{pmatrix} 1 \\ 1 \end{pmatrix}\).

\end{solution}

\begin{problem}{2}
    Use R and the updated Darwin and Tahiti standardized SLP data to reproduce the EOFs and PCs and to plot the EOF pattern maps and PC time series.
\end{problem}

\begin{solution}
    R code:
    \begin{lstlisting}[language=R]
setwd("C:/Users/sebas/OneDrive/Desktop/Homework/LinAlg")
Pda<-read.table("PSTANDdarwin.txt", header=F)
dim(Pda) 
Pda
pdaDec<-Pda[,13] #Darwin Dec standardized SLP anomalies data
pdaDec

Pta<-read.table("PSTANDtahiti.txt", header=F)
ptaDec=Pta[,13] #Tahiti Dec standardized SLP anomalies
ptada1 = rbind(pdaDec, ptaDec) #space-time data matrix
ptada1
#Space-time data format
colnames(ptada1) <- 1951:2015

rownames(ptada1)<-c("Darwin", "Tahiti")
ptada1

dim(ptada1)
svdptd = svd(ptada1)

svdptd

U=round(svdptd$u, digits=2)
U

D = round(diag(svdptd$d), digits = 2)
D

V =round(svdptd$v, digits=2)
t(V)

eof1 = U[, 1]
eof2 = U[, 2]

PC1 = V[, 1]
PC2 = V[, 2]


x = c(eof1[1], eof2[1])
y = c(eof1[2], eof2[2])

# Plot EOFs with different colors
plot(x,y, col = c("blue", "red"), pch = 16,)

years = 1951:2015

# Plot PC over time
plot(years, PC1, type = 'l', col = 'blue', xlab = 'Year', ylab = 'Principal Component Value', 
     main = 'Principal Components over Time', ylim = range(c(PC1, PC2)))
lines(years, PC2, type = 'l', col = 'red')
legend('topright', legend = c('PC1', 'PC2'), col = c('blue', 'red'), lty = 1)
    \end{lstlisting}
    Plots:
    \begin{figure}[H]
        \centering
        \includegraphics[width=0.5\textwidth]{coloreof.png}
        \caption{EOFs}
    \end{figure}
    \begin{figure}[H]
        \centering
        \includegraphics[width=0.5\textwidth]{DarTihPC.png}
        \caption{PCs}
    \end{figure}
\end{solution}

\begin{problem}{3}
    \begin{enumerate}[label=(\alph*)]
        \item A covariance matrix \( C \) can be computed from a space-time observed anomaly data matrix \( X \), which has \( N \) rows for spatial locations and \( Y \) columns for time in years:
        \[
            C = \frac{X \cdot X^T}{Y}
        \]
        This results in an \( N \times N \) matrix. Select an \( X \) data matrix from the USHCN annual total precipitation data at three California stations, ordered from north to south: San Francisco, CA (040693); Santa Barbara, CA (047902); and San Diego, CA (042239). Use data from five years, spanning 2001 to 2005. Consider the anomaly data relative to the 2001-2005 mean and employ R to calculate the covariance matrix with \( N = 3 \) and \( Y = 5 \).

        \item Utilize R to determine the inverse matrix of the covariance matrix \( C \).

        \item Use R to compute the eigenvalues and eigenvectors of \( C \).

        \item Perform Singular Value Decomposition (SVD) on the data matrix \( X \) using R, such that \( X = UDV^T \). Explicitly present the three resulting matrices \( U \), \( D \), and \( V \).

        \item Explore the relationship between the eigenvalues of \( C \) and the matrix \( D \) using R.

        \item Compare the eigenvectors of \( C \) with the matrix \( U \).

        \item Plot the Principal Component (PC) time series and provide a description of their behavior.
    \end{enumerate}
\end{problem}

\begin{solution}
    \begin{enumerate}[label=(\alph*)]
        \item The covariance matrix $C$:
            \[
            C = \begin{pmatrix}
                0.8000000 & 0.3838680 & 0.7204883 \\
                0.3838680 & 0.8000000 & 0.2709303 \\
                0.7204883 & 0.2709303 & 0.8000000
            \end{pmatrix}
            \]
            R code:
            \begin{lstlisting}[language=R]
                SF = c(214.9, 107.7, 233.4, 337.5, 359.1)
                SB = c(695.9, 389.7, 391.2, 446.8, 768.3)
                SD = c(496.2, 475.1, 490.7, 685.7, 664.2)
                
                mSF = mean(SF)
                mSB = mean(SB)
                mSD = mean(SD)
                
                aSF = (SF - mSF)/sd(SF)
                aSB = (SB - mSB)/sd(SB)
                aSD = (SD - mSD)/sd(SD)
                
                A = rbind(aSF, aSB, aSD)
                
                C = A %*% t(A) / 5
                
                C
            \end{lstlisting}
        
        \item The inverse of the covariance matrix $C^{-1}$:
            \[
            C^{-1} = \begin{pmatrix}
                8.097171 & -1.5990410 & -6.7508602 \\
                -1.599041 & 1.7277197 & 0.8549985 \\
                -6.750860 & 0.8549985 & 7.0403385
            \end{pmatrix}
            \]
            R code:
            \begin{lstlisting}[language=R]
                invC = solve(C)

                invC
            \end{lstlisting}
        

        \item Eigenvalues and eigenvectors of $C$:
            \begin{itemize}
                \item Eigenvalues: $\lambda_1 = 1.74763214$, $\lambda_2 = 0.58378101$, $\lambda_3 = 0.06858685$
                \item Eigenvectors:
                    \[
                    \begin{pmatrix}
                        0.6495950 & 0.2108313 & 0.7304632 \\
                        0.4403361 & -0.8875627 & -0.1354128 \\
                        0.6197826 & 0.4096128 & -0.6693929
                    \end{pmatrix}
                    \]
            \end{itemize}
            R code:
            \begin{lstlisting}[language=R]
                eigC = eigen(C)

                eigC
            \end{lstlisting}
        
        \item SVD of $C$:
        \[
            \begin{aligned}
                D &= \begin{pmatrix}
                    2.9560380  & 0 & 0 \\
                    0 & 1.7084803  & 0 \\
                    0 & 0 & 0.5856059
                \end{pmatrix}\\
                U &= \begin{pmatrix}
                    -0.6495950 & 0.2108313 & -0.7304632 \\
                    -0.4403361 & -0.8875627 & 0.1354128 \\
                    -0.6197826 & 0.4096128 & 0.6693929
                \end{pmatrix}\\
                V &= \begin{pmatrix}
                    0.08105994 & -0.65102219 & -0.09272693 \\
                    0.60884394 & 0.05280542 & 0.59658302 \\
                    0.30415262 & 0.23730478 & -0.77171237 \\
                    -0.36257497 & 0.65581870 & 0.17907751 \\
                    -0.63148153 & -0.29490670 & 0.08877877
                \end{pmatrix}
            \end{aligned}
        \]
        R code:
        \begin{lstlisting}[language=R]
            X = svd(A)

            X
        \end{lstlisting}
        \item The relationship between the eigenvalues of $C$ and the matrix $D$ is $\lambda_i = d_i^2/5$.
        R code:
        \begin{lstlisting}[language=R]
            2.9560380^2/5
            1.7084803^2/5
            0.5856059^2/5
        \end{lstlisting}
        \item The eigenvectors of $C$ are the same as the columns of $U$, but some of the signs are flipped.
        R code:
        \begin{lstlisting}[language=R]
            eigC$vectors
            X$u
        \end{lstlisting}
        \item The PC time series R code:
        \begin{lstlisting}[language=R]
            pc1 = X$v[,1]
            pc2 = X$v[,2]
            pc3 = X$v[,3]

            years = 2001:2005

            plot(years, pc1, type='b', col='red', ylim=range(c(pc1,pc2,pc3)), 
            xlab='Year', ylab='PC Value', main='Principal Component Time Series')
            lines(years, pc2, col='blue')
            lines(years, pc3, col='green')
            legend('topright', legend=c('PC1', 'PC2', 'PC3'), 
            col=c('red', 'blue', 'green'), lty=1, pch=1)
        \end{lstlisting}
        \begin{figure}[H]
            \centering
            \includegraphics[width=0.5\textwidth]{PCtimeof3cities.png}
            \caption{PC Time Series}
        \end{figure}
        PC1 is San Francisco, PC2 is Santa Barbara, PC3 is San Diego. They all vary in different ways. There does not seem to be any connection between the three.
    \end{enumerate}
\end{solution}

\begin{problem}{4}
    The burning of propane \(C_3H_8\) with oxygen \(O_2\) produces water \(H_2O\) and carbon dioxide \(CO_2\). Balance the chemical reaction equation.
\end{problem}

\begin{solution}
    Write the equation:
    \[
    C_3H_8 + O_2 \rightarrow CO_2 + H_2O
    \]
    Write the equain terms of \(x, y, z\):
    \[
    C_3H_8 + xO_2 = yCO_2 + zH_2O
    \]
    Balance the equation:
    \[
        \begin{aligned}
            C&: 3 = y \\
            H&: 8 = 2z \\
            O&: 2x = 2y + z
        \end{aligned}
    \]
    Solve the equations:
    \[
    y = 3, \quad z = 4, \quad x = 5
    \]
    Therefore, the balanced equation is:
    \[
    C_3H_8 + 5O_2 = 3CO_2 + 4H_2O
    \]
\end{solution}

\begin{problem}{5}
    Write a computer code to
    \begin{enumerate}[label=(\alph*)]
        \item Read the NOAAGlobalTemp data file, and
        \item Generate a \(4 \times 8\) space-time data matrix for the December mean surface air temperature anomaly data of four grid boxes and eight years.
    \end{enumerate}
\end{problem}

\begin{solution}
    Rcode:
    \begin{lstlisting}[language=R]
setwd("C:/Users/sebas/OneDrive/Desktop/Homework/LinAlg")

noaa_data <- read.csv("NOAAGlobalT.csv", header = TRUE)

noaa1 = noaa_data[1777,]
noaa2 = noaa_data[1778,]
noaa3 = noaa_data[1779,]
noaa4 = noaa_data[1780,]

# Adjust DecDat to use data from 2001 to 2008
DecIndex = seq(1467, 1551, 12)
DecDat = noaa1[DecIndex]
DecDat2 = noaa2[DecIndex]
DecDat3 = noaa3[DecIndex]
DecDat4 = noaa4[DecIndex]

# Verify the data
DecDat

#rbind some stuff
Xm = rbind(DecDat, DecDat2, DecDat3, DecDat4)

rownames(Xm) = c('Lat 237.5/Lon 32.5', 'Lat 242.5/Lon 32.5', 'Lat 247.5/Lon 32.5', 'Lat 252.5/Lon 32.5')
colnames(Xm) = 2001:2008
Xm
    \end{lstlisting}
    R code output:
    \begin{lstlisting}[language=R]
        2001    2002   2003    2004   2005    2006    2007
Lat 237.5/Lon 32.5 -0.4677  0.1699 0.1579 -0.0300 0.3653  0.3606 -1.0685
Lat 242.5/Lon 32.5 -0.8994 -0.2796 0.5777  0.2757 0.9472 -0.1739 -0.9703
Lat 247.5/Lon 32.5 -0.8168 -0.6909 0.4853 -0.1080 0.7542 -0.4942  0.0847
Lat 252.5/Lon 32.5  0.8871  0.1945 1.5203  0.2909 0.3835 -0.0018  0.7548
                      2008
Lat 237.5/Lon 32.5 -0.1625
Lat 242.5/Lon 32.5  0.5768
Lat 247.5/Lon 32.5  1.5718
Lat 252.5/Lon 32.5  0.5495
    \end{lstlisting}
\end{solution}

\begin{problem}{6}
    Write a computer code to find the inverse of the following matrix
    \[
        \begin{pmatrix}
            1.7 & -0.7 & 1.3 \\
            -1.6 & -1.4 & 0.4 \\
            -1.5 & -0.3 & 0.6
        \end{pmatrix}
    \]
\end{problem}

\begin{solution}
    Rcode:
    \begin{lstlisting}[language=R]
    A1x = matrix(c(1.7,-0.7,1.3,-1.6,-1.4,0.4,-1.5,-0.3,0.6), nrow=3, byrow=TRUE)

    solve(A1x)
    \end{lstlisting}
    Inverse of the matrix:
    \[
    \begin{pmatrix}
        0.2010050 & -0.008375209 & -0.4299274 \\
        -0.1005025 & -0.829145729 & 0.7705193 \\
        0.4522613 & -0.435510888 & 0.9771078
    \end{pmatrix}
    \]
\end{solution}

\begin{problem}{7}
    Write a computer code to solve the following linear system of equations
    \[
        Ax = b,
    \]
    where
    \[
        A = \begin{pmatrix}
            1 & 2 & 3 \\
            4 & 5 & 6 \\
            7 & 8 & 0
        \end{pmatrix}
    \]
    \[
        x = \begin{pmatrix}
            x1 \\
            x2 \\
            x3
        \end{pmatrix}
    \]
    \[
        b = \begin{pmatrix}
            -1 \\
            0 \\
            1
        \end{pmatrix}
    \]
\end{problem}

\begin{solution}
    Rcode:
    \begin{lstlisting}[language=R]
        A2x = matrix(c(1, 2, 3, 4, 5, 6, 7, 8, 0), nrow=3, byrow=TRUE)
        b2 = c(-1, 0, 1)
        solve(A2x, b2)
    \end{lstlisting}
\end{solution}

\begin{problem}{8}
    The following equation
    \[
        \begin{pmatrix}
            1 & 2 & 3 \\
            4 & 5 & 6 \\
            7 & 8 & 9
        \end{pmatrix}
        \begin{pmatrix}
            x_1 \\
            x_2 \\
            x_3
        \end{pmatrix}
        =
        \begin{pmatrix}
            0 \\
            0 \\
            0
        \end{pmatrix}
    \]
    has infinitely many solutions, and cannot be directly solved by a simple com-
    puter command, such as solve(A, b).
    \begin{enumerate}[label=(\alph*)]
        \item Show that the three row vectors of the coefficient matrix are not linearly
        independent.
        \item Because of the dependence, the linear system has only two independent
        equations. Thus, reduce the linear system into two equations by treating \(x_3\)
        as an arbitrary value while treating \(x_1\) and \(x_2\) as variables.
        \item Solve the two equations for \(x_1\) and \(x_2\) and express them in terms of \(x_3\).
    \end{enumerate}
\end{problem}

\begin{solution}
    \[
        \begin{aligned}
            x_1 + 2x_2 + 3x_3 &= 0 \\
            4x_1 + 5x_2 + 6x_3 &= 0 \\
            7x_1 + 8x_2 + 9x_3 &= 0
        \end{aligned}
    \]
    \[
        2(4x_1 + 5x_2 + 6x_3) - (x_1 + 2x_2 + 3x_3) = 7x_1 + 8x_2 + 9x_3
    \]
    Therefore, the row vectors are not linearly independent.

    Now, let $x_3 = t$. Then, the system of equations becomes:
    \[
        \begin{aligned}
            x_1 + 2x_2 + 3t &= 0 \\
            4x_1 + 5x_2 + 6t &= 0
        \end{aligned}
    \]
    Solve for $x_1$ and $x_2$ in terms of $t$:
    \[
        \begin{aligned}
            x_1 &= -2x_2 - 3t \\
            x_2 &= -4x_1 - 6t
        \end{aligned}
    \]
    Then
    \[
        \begin{aligned}
            x_2 &= -4(-2x_2 - 3t) - 6t \\
            &= 8x_2 + 12t - 6t \\
            &= 8x_2 + 6t
            -7x_2 &= 6t \\
            x_2 &= -\frac{6}{7}t
        \end{aligned}
    \]
    So,
    \[
        \begin{aligned}
            x_1 &= -2x_2 - 3t \\
            &= -2(-\frac{6}{7}t) - 3t \\
            &= \frac{12}{7}t - 3t \\
            &= -\frac{9}{7}t
        \end{aligned}
    \]


\end{solution}

\begin{problem}{9}
    Ethane is a gas similar to the greenhouse gas methane and can burn with oxygen to form carbon dioxide and water:
    \[
        C_2H_6 + O_2 \rightarrow CO_2 + H_2O
    \]
    Given two ethane molecules, how many molecules of oxygen, carbon dioxide and water should be in order for this chemical reaction equation to be balanced?
\end{problem}

\begin{solution}
    Assign:
    \[
        x : O_2 \quad y : CO_2, \quad z : H_2O
    \]
    Write the equation:
    \[
        2C_2H_6 + xO_2 = yCO_2 + zH_2O
    \]
    Balance the equation:
    \[
        \begin{aligned}
            C&: 2 \cdot 2 = y \\
            H&: 2 \cdot 6 = 2z \\
            O&: 2x = 2y + z
        \end{aligned}
    \]
    Set up a matrix to find $O$:
    \[
        \begin{pmatrix}
            0 & 1 & 0 \\
            0 & 0 & 1 \\
            2 & -2 & -1
        \end{pmatrix}
        \begin{pmatrix}
            4 \\
            6 \\
            0
        \end{pmatrix}
    \]
    Put this into R and solve:
    \begin{lstlisting}[language=R]
        A3 = matrix(c(0, 1, 0, 0, 0, 1, 2, -2, -1), nrow=3, byrow=TRUE)
        b3 = c(4, 6, 0)
        solve(A3, b3)
    \end{lstlisting}
    The solution is:
    \[
        \begin{pmatrix}
            7 \\
            4 \\
            6
        \end{pmatrix}
    \]
    So, the solution is:
    \[
        2C_2H_6 + 7O_2 = 4CO_2 + 6H_2O
    \]
\end{solution}

\begin{problem}{10}
    \begin{enumerate}[label=(\alph*)]
        \item Use matrix multiplication to show that the vector
        \[
            u = \begin{pmatrix}
                1 \\
                1
            \end{pmatrix}
        \]
        is not an eigenvector of the following matrix
        \[
            A = \begin{pmatrix}
                0 & 4 \\
                -2 & -7
            \end{pmatrix}
        \]
        \item Find all the unit eigenvectors of matrix A in Part (a)
    \end{enumerate}
\end{problem}

\begin{solution}
    \begin{enumerate}[label=(\alph*)]
        \item Multiply $A$ by $u$:
        \[
            A \begin{pmatrix}
                1 \\
                1
            \end{pmatrix} = \begin{pmatrix}
                0 & 4 \\
                -2 & -7
            \end{pmatrix} \begin{pmatrix}
                1 \\
                1
            \end{pmatrix} = \begin{pmatrix}
                4 \\
                -9
            \end{pmatrix}
        \]
        If you try scaling $u$ by \(4\), you get:
        \[
        4 \begin{pmatrix}
                1 \\
                1
            \end{pmatrix} = \begin{pmatrix}
                4 \\
                4
            \end{pmatrix}
        \]
        This is not \(\begin{pmatrix}
                4 \\
                -9
            \end{pmatrix}
        \).
        \item Use R to find the unit eigenvectors of $A$:
        \begin{lstlisting}[language=R]
            A = matrix(c(0, 4, -2, -7), nrow=2, byrow=TRUE)
            eigen_result = eigen(A)
            eigenvectors = eigen_result$vectors
            unit_eigenvectors = apply(eigenvectors, 2, function(v) v / sqrt(sum(v^2)))
            unit_eigenvectors
        \end{lstlisting}
        The unit eigenvectors are:
        \[
            \begin{pmatrix}
                -0.5838895 \\
                0.8118331
            \end{pmatrix}, \quad
            \begin{pmatrix}
                0.9410038 \\
                -0.3383961
            \end{pmatrix}
        \]
    \end{enumerate}
\end{solution}

\end{document}
