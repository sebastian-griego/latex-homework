\documentclass[12pt]{article}
\usepackage[utf8]{inputenc}
\usepackage{amsmath, amssymb, amsthm}
\usepackage{enumitem}
\usepackage{geometry}
\usepackage{fancyhdr}

% Page setup
\geometry{letterpaper, margin=1in}
\setlength{\parindent}{0pt}
\setlength{\parskip}{1em}
\pagestyle{fancy}
\fancyhf{}
\fancyhead[L]{\textbf{Sebastian Griego}}  % Replace with your name
\fancyhead[C]{\textbf{ODEs}}  % Replace with your course name
\fancyhead[R]{\textbf{Assignment \#1}}  % Replace with your assignment number
\fancyfoot[C]{\thepage}

\newenvironment{problem}[1]{
    \textbf{Problem #1:}
}{
    \rmfamily \vspace{1em}
}

\newenvironment{solution}{
    \textbf{Solution:}
    
}{
    
    \vspace{2em}
}

\begin{document}

\title{Homework \#1}  % Replace with the homework number
\author{Sebastian Griego}  % Replace with your name
\maketitle

\begin{problem}{1}
    \[
        \begin{aligned} 
            \text{Solve}\\
            &\frac{dy}{dx}+2xy=f(x),\quad y(0)=2 \\
            \text{where} \\
            &f(x)=\begin{cases}
                x, \quad 0\leq x\leq 1\\
                0, \quad x>1
            \end{cases}
        \end{aligned}
    \]
\end{problem}


\begin{solution}
    \[
        \frac{dy}{dx}+g(x)y=f(x),\quad y(0)=y_0\\
    \]
Define the integrating factor:
    \[
        \begin{aligned}
            I(x,0)&=e^{\int_{0}^x g(u)du}\\
            &=e^{\int_{0}^x 2udu}\\
            &=e^{x^2-0^2} \\
            &=e^{x^2}
        \end{aligned}
    \]
Evaluate:
    \[
        \begin{aligned} 
            \frac{dI}{dx} &= e^{\int_{0}^xg(u)du} \cdot \frac{d}{dx}\int_{0}^{x}g(u)du\\
            &= e^{\int_{0}^x g(u)du}\cdot g(x)\\
            &= g(x)I\\
        \end{aligned}
    \]
Multiply the differential equation by the integrating factor:
    \[
        \begin{aligned}
            I\frac{dy}{dx}+Ig(x)y &= If(x)\\
            I\frac{dy}{dx} + \frac{dI}{dt} \cdot y &= f(x)I\\
            \frac{d}{dx}(I\cdot y) &= If(x)\\
        \end{aligned}
    \]
Integrate both sides:
    \[
        \begin{aligned}
            \int_{0}^x \frac{d}{du}(I(u,0)\cdot y)du &= \int_{0}^x I(u,0)f(u)du\\
            I(u,0)y(u)\Big|_{0}^x &= \int_{0}^x I(u,0)f(u)du\\
            I(x,0)y(x)-I(0,0)y(0) &= \int_{0}^x I(u,0)f(u)du\\
            I(x,0)y(x)- 1 \cdot y_0 &= \int_{0}^x I(u,0)f(u)du\\
            I(x,0)y(x)- y_0 &= \int_{0}^x I(u,0)f(u)du\\
        \end{aligned}
    \]
Plug in values:
    \[
        \begin{aligned}
            I(x,0)y(x)- y_0 &= \int_{0}^x I(u,0)f(u)du\\
            e^{x^2}y(x)- 2 &= \int_{0}^x e^{u^2} f(u)du\\
        \end{aligned}
    \]  
Consider \(f(u)\):
    \[
        \begin{aligned}
                f(u) &= \begin{cases}
                u, \quad 0\leq u\leq 1\\
                0, \quad u>1
                \end{cases}\\
        \end{aligned}
    \]
The integral is from \(0\) to \(x\), and \(u\) is between these values:
    \[
        \begin{aligned}
            e^{x^2}y(x)- 2 &= \int_{0}^x e^{u^2} f(u)du\\
            &= \begin{cases}
                \int_{0}^x u \cdot e^{u^2}  du, & 0\leq x\leq 1\\[0.5em]
                \int_{0}^1 u \cdot e^{u^2}  du, & x>1\\[0.5em]
            \end{cases}
        \end{aligned}
    \]
Evaluate the integrals:
    \[
        \begin{aligned}
            \int_{0}^x u \cdot e^{u^2}  du &= \frac{1}{2} e^{u^2}\Big|_{0}^x\\
            &= \frac{1}{2} e^{x^2}-\frac{1}{2} e^{0^2}\\
            &= \frac{1}{2} e^{x^2}-\frac{1}{2}\\
            &= \frac{e^{x^2}-1}{2}
        \end{aligned}
    \]
    and
    \[
        \begin{aligned}
            \int_{0}^1 u \cdot e^{u^2}  du &= \frac{1}{2} e^{u^2}\Big|_{0}^1\\
            &= \frac{1}{2} e^{1^2}-\frac{1}{2} e^{0^2}\\
            &= \frac{1}{2} e - \frac{1}{2}\\
            &= \frac{e-1}{2}
        \end{aligned}
    \]
Solve for \(y(x)\) when \(0\leq x\leq 1\):
    \[
        \begin{aligned}
            e^{x^2}y(x)- 2 &= \frac{e^{x^2}-1}{2}\\
            e^{x^2}y(x) &= 2 + \frac{e^{x^2}-1}{2}\\
            y(x) &= \frac{2}{e^{x^2}} + \frac{e^{x^2}-1}{2e^{x^2}}\\
            &= \frac{4}{2e^{x^2}} + \frac{1}{2} - \frac{1}{2e^{x^2}}\\
            &= \frac{3}{2e^{x^2}} + \frac{1}{2}
        \end{aligned}
    \]
Solve for \(y(x)\) when \(x>1\):
    \[
        \begin{aligned}
            e^{x^2}y(x)- 2 &= \frac{e-1}{2}\\
            e^{x^2}y(x) &= 2 + \frac{e-1}{2}\\
            y(x) &= \frac{2}{e^{x^2}} + \frac{e-1}{2e^{x^2}}\\
            &= \frac{4}{2e^{x^2}} + \frac{e-1}{2e^{x^2}}\\
            &= \frac{e+3}{2e^{x^2}}
        \end{aligned}
    \]
The final solution can be written as:
    \[
        y(x) = \begin{cases}
            \frac{3}{2e^{x^2}} + \frac{1}{2}, & 0 \leq x \leq 1 \\[0.5em]
            \frac{e+3}{2e^{x^2}}, & x > 1
        \end{cases}
    \]
\end{solution}


\begin{problem}{2}
Write a first order, linear, inhomogenous differential equation whose
solution \(y(t)\) goes to 4 as \(t \to \infty\).
\end{problem}

\begin{solution}
Consider a general first order linear inhomogenous differential equation:
    \[
        \frac{dy}{dt} + g(t)y = f(t) \quad y(t_0) = y_0
    \]
This has a general solution of:
    \[
        y(t) = e^{-\int_{t_0}^t g(s)ds} \cdot y_0 + \int_{t_0}^t f(u)e^{-\int_{u}^t g(s)ds}du
    \]
Define:
\[
    \begin{aligned}
        y_0 &= 4\\
        f(t) &= 4t\\
        g(t) &= t\\
    \end{aligned}
\]
Plug into the general solution:
    \[
        \begin{aligned}
            y(t) &= e^{-\int_{t_0}^t s ds} \cdot 4 + \int_{t_0}^t 4te^{-\int_{u}^t s ds}du\\
            &= 4 + \int_{t_0}^t 4te^{\frac{-(t^2 - u^2)}{2}}du\\
            &= 4 + (4 - 4e^{\frac{-(t^2 - t_0^2)}{2}})
        \end{aligned}
    \]
Take the limit as \(t \to \infty\):
    \[
        \begin{aligned}
            \lim_{t \to \infty} y(t) &= \lim_{t \to \infty} 4 + (4 - 4e^{\frac{-(t^2 - t_0^2)}{2}})\\
            &= 4 + 4 - 4e^0\\
            &= 8 - 4\\
            &= 4
        \end{aligned}
    \]
Therefore, the differential equation is:
    \[
        \frac{dy}{dt} + ty = 4t \quad y(0) = 4
    \]
\end{solution}

\begin{problem}{3}
    Find the general solution to
    \[
        \dot{x} = x + \cos(t)
    \]
    Show by choosing the initial condition appropriately that there is exactly one periodic solution to this problem. Remember, by periodic
we mean that there is some \(T\) such that \(x(t + T) = x(t)\).
\end{problem}

\begin{solution}
Assume the initial condition:
    \[
        x_0 = 0
    \]
Rewrite the differential equation:
    \[
        \dot{x} - x = \cos(t)
    \]
Response:
    \[
        \begin{aligned}
            x(t) &= \int_{t_0}^t \cos(s)e^{(t-s)}ds\\
            &= e^{t}\int_{t_0}^t \cos(s)e^{-s}ds\\
            &= e^{t}\left(-\cos(s)e^{-s}\Big|_{t_0}^t-\int_{t_0}^t \sin(s)e^{-s}ds\right)\\
            &= e^{t}\left((-\cos(s)e^{-s}+\sin(s)e^{-s})\Big|_{t_0}^t - \int_{t_0}^t \cos(s)e^{-s}ds\right)\\
            &= e^{t}\left(-\cos(s)e^{-s}+\sin(s)e^{-s}\right)\Big|_{t_0}^t - x(t)\\
            2x(t) &= e^{t}\left(-\cos(s)e^{-s}+\sin(s)e^{-s}\right)\Big|_{t_0}^t\\
            x(t) &= \frac{1}{2} e^{t}\left(-\cos(s)e^{-s}+\sin(s)e^{-s}\right)\Big|_{t_0}^t\\
            &= \frac{1}{2} e^{t} \left(-e^{-t}\cos(t)+e^{-t}\sin(t)+e^{-t_0}\cos(t_0)-e^{-t_0}\sin(t_0) \right)\\
            &= \frac{1}{2} \left(-\cos(t) + \sin(t) + e^{(t-t_0)}(\cos(t_0)-\sin(t_0))\right)\\
            &= \frac{1}{2} \left(-\cos(t) + \sin(t)\right)+ \frac{1}{2}e^{(t-t_0)} \left(\cos(t_0)-\sin(t_0)\right)\\
            &= \frac{-1}{\sqrt{2}} \left(\frac{1}{\sqrt{2}} \cos(t) - \frac{1}{\sqrt{2}}\sin(t)\right) + \frac{1}{2}e^{(t-t_0)} \left(\cos(t_0)-\sin(t_0)\right)\\
            &= \frac{-1}{\sqrt{2}} \left(\cos(\frac{\pi}{4}) \cdot \cos(t) - \sin(\frac{\pi}{4}) \cdot \sin(t)\right) + \frac{1}{2}e^{(t-t_0)} \left(\cos(t_0)-\sin(t_0)\right)\\
            &= \frac{-1}{\sqrt{2}} \left(\cos(t-\frac{\pi}{4})\right) + \frac{1}{2}e^{(t-t_0)} \left(\cos(t_0)-\sin(t_0)\right)\\
        \end{aligned}
    \]
With the initial condition \(t_0 = \frac{\pi}{4}\), the solution becomes:
    \[
        \begin{aligned}
            x(t) &= \frac{-1}{\sqrt{2}} \left(\cos(t-\frac{\pi}{4})\right) + \frac{1}{2}e^{(t-\frac{\pi}{4})} \left(\cos(\frac{\pi}{4})-\sin(\frac{\pi}{4})\right)\\
            &= \frac{-1}{\sqrt{2}} \left(\cos(t-\frac{\pi}{4})\right) + \frac{1}{2}e^{(t-\frac{\pi}{4})} \left(\frac{1}{\sqrt{2}}-\frac{1}{\sqrt{2}}\right)\\
            &= \frac{-1}{\sqrt{2}} \left(\cos(t-\frac{\pi}{4})\right)
        \end{aligned}
    \]
This solution is periodic with period \(2\pi\).
\end{solution}

\begin{problem}{4}
    Consider the equation
        \[
            \dot{x} + p(t)x = 0
        \]
    Suppose that \(p(t)\) is continuous  and periodic with period \(T\), i.e. \(p(t + T) = p(t)\). Show that the solution \(x(t)\) for any initial condition is
    periodic if and only if
    \[
        \int_{0}^T p(s)ds = 0
    \]
    Said another way, you are showing that if \(p(t)\) has zero average in time, then the solution will be periodic.
\end{problem}

\begin{solution}
The general solution to the equation $\dot{x} + p(t)x = 0$ is:
    \[
        x(t) = x_0 e^{-\int_{t_0}^t p(s)ds}
    \]
Suppose the solution is periodic with period \(T\) for any initial condition:
    \[
        x(t+T) = x(t) \quad \forall t, x_0
    \]
Use this equality to simplify the general solution:
    \[
        \begin{aligned}
            x(t+T) &= x(t)\\
            x_0 e^{-\int_{t_0}^{t+T} p(s)ds} &= x_0 e^{-\int_{t_0}^t p(s)ds}\\
            e^{-\int_{t_0}^{t+T} p(s)ds} &= e^{-\int_{t_0}^t p(s)ds}\\
            -\int_{t_0}^{t+T} p(s)ds &= -\int_{t_0}^t p(s)ds\\
            -\int_{t_0}^t p(s)ds - \int_{t}^{t+T} p(s)ds &= -\int_{t_0}^{t} p(s)ds\\
            -\int_{t}^{t+T} p(s)ds &= 0\\
            \int_{t}^{t+T} p(s)ds &= 0
        \end{aligned}
    \]
Use the fact that \(p(t)\) is periodic with period \(T\). The periodic function is being integrated over the length of the entire period, so the value of \(t\) is irrelevant:
    \[
        \begin{aligned}
            \frac{d}{dt}\int_{t}^{t+T} p(s)ds &= p(t+T)-p(t) = 0\\
            \implies \int_{t}^{t+T} p(s)ds &= c\\
            \implies \int_{0}^{T} p(s)ds &= \int_{t}^{t+T} p(s)ds = 0
        \end{aligned}
    \]
This proves \(\implies\).

Now, assume:
    \[
        \int_{0}^T p(s)ds = 0
    \]
We are given that \(p(t)\) is periodic with period \(T\). We can use this to show that the solution is periodic:
    \[
        \begin{aligned}
            x(t+T) &= x_0 e^{-\int_{t_0}^{t+T} p(s)ds}\\
            &= x_0 e^{-\int_{t_0}^{t} p(s)ds - \int_{t}^{t+T} p(s)ds} \\
            &= x_0 e^{-\int_{t_0}^{t} p(s)ds - \int_{0}^{T} p(s)ds} \quad \text{as shown above.}\\
            &= x_0 e^{-\int_{t_0}^{t} p(s)ds-0}\\
            &= x_0 e^{-\int_{t_0}^{t} p(s)  ds}\\
            x(t+T) &= x(t)
        \end{aligned}
    \]
This proves \(\impliedby\).
\end{solution}

\begin{problem}{5}
For the system in the prior problem, show that if
    \[
        \int_{0}^T p(s)ds = 0
    \]
then the solution is uniformly stable. Note, you’ll need to use the fact that a continuous function, which is \(x(t)\) in this case, is bounded, i.e. \(\exists M > 0\) such that \(|x(t)| \leq M\), over a finite interval.
\end{problem}

\begin{solution}
Assume:
    \[
        \int_{0}^T p(s)ds = 0
    \]
From the previous problem, we know this implies the solution \(x(t)\) is periodic with period \(T\) for any initial condition \(x_0\). Since \(x(t)\) is continuous on the closed, bounded interval \([0, T]\), \(x(t)\) must have a max and min by the Extreme Value Theorem. Define:
    \[
        M = \max |x(t)| \quad  t \in [0, T]
    \]
    
\(x(t)\) is periodic with period \(T\), so:
    
    \[
        |x(t)| \leq M \quad \forall t
    \]
Now, define \(\gamma(t_0)\) such that
    \[
        \begin{aligned}
            \gamma(t_0) &= \frac{M}{|x_0|}\\
            M &= \gamma(t_0) |x_0|\\
        \end{aligned}
    \]
Both \(M > 0\) and \(|y_0| > 0\), so \(\gamma(t_0) > 0\).

We now have:
    \[
        \begin{aligned}
            |x(t)| &\leq M \quad \forall t \geq 0\\
            |x(t)| &\leq \gamma(t_0) |x_0| \quad \forall t \geq t_0
        \end{aligned}
    \]
The system is uniformly stable:
    \[
        \begin{aligned}
            \text{Remember } &x(t) = x_0 e^{-\int_{t_0}^{t} p(s)  ds}\\
            \text{Let } &p(t) \geq 0\\
            \implies &\int_{t_0}^t p(s)ds \geq 0\\
            \implies &e^{-\int_{t_0}^t p(s)ds} \leq 1\\
            \implies &|x(t)| \leq |x_0|
        \end{aligned}
    \]
Therefore, \(x(t)\) is uniformly stable.


\end{solution}


\end{document}
