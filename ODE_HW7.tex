\documentclass[12pt]{article}
\usepackage[utf8]{inputenc}
\usepackage{amsmath, amssymb, amsthm}
\usepackage{enumitem}
\usepackage{geometry}
\usepackage{fancyhdr}
\usepackage{pgfplots}
\usepackage{tikz}
\usepackage{float}
\usepackage{graphicx}
\DeclareMathOperator{\Tr}{Tr}
\DeclareMathOperator{\rng}{rng}
\DeclareMathOperator{\norm}{||}
\DeclareMathOperator{\NN}{\mathbb{N}}
\DeclareMathOperator{\ZZ}{\mathbb{Z}}
\DeclareMathOperator{\QQ}{\mathbb{Q}}
\DeclareMathOperator{\RR}{\mathbb{R}}
\DeclareMathOperator{\CC}{\mathbb{C}}

% Page setup
\setlength{\headheight}{15pt}
\geometry{letterpaper, margin=1in}
\setlength{\parindent}{0pt}
\setlength{\parskip}{1em}
\pagestyle{fancy}
\fancyhf{}
\fancyhead[L]{\textbf{Sebastian Griego}}  % Replace with your name
\fancyhead[C]{\textbf{ODEs}}  % Replace with your course name
\fancyhead[R]{\textbf{Assignment \#7}}  % Replace with your assignment number
\fancyfoot[C]{\thepage}

\newenvironment{problem}[1]{
    \textbf{Problem #1:}
}{
    \rmfamily \vspace{1em}
}

\newenvironment{solution}{
    \textbf{Solution:}
    
}{
    
    \vspace{2em}
}

\begin{document}

\title{Homework \#7}  % Replace with the homework number
\author{Sebastian Griego}  % Replace with your name
\maketitle


\begin{problem}{1}
    Given a transfer function \(\hat{G}(s),\) let \((\tilde{A}, \tilde{B}, \tilde{C}, \tilde{D})\) be a realization
    for its transpose \(\tilde{G}(s) := \hat{G}(a)^T\). Show that \((A, B, C, D)\) is a realization for \(\hat{G}(s)\) with
    \[
        A = \tilde{A}^T, \quad B = \tilde{C}^T, \quad C = \tilde{B}^T, \quad D = \tilde{D}^T
    \]
\end{problem}

\begin{solution}
    \((\tilde{A}, \tilde{B}, \tilde{C}, \tilde{D})\) is a realization for \(\tilde{G}(s) := \hat{G}(s)^T\). So
    \[
        \begin{aligned}
            \hat{G}(s)^T &= \tilde{C}(s - \tilde{A})^{-1}\tilde{B} + \tilde{D} \\
            (\hat{G}(s)^T)^T &= (\tilde{C}(s - \tilde{A})^{-1}\tilde{B} + \tilde{D})^T \\
            \hat{G}(s) &= (\tilde{C}(s - \tilde{A})^{-1}\tilde{B})^T + \tilde{D}^T \\
            \hat{G}(s) &= \tilde{B}^T(s - \tilde{A}^T)^{-1}\tilde{C}^T + \tilde{D}^T, \quad sI\text{ is diagonal so the transpose does not change anything} \\
            \hat{G}(s) &= C(s - A)^{-1}B + D
        \end{aligned}
    \]
    
\end{solution}

\newpage

\begin{problem}{2}
    \begin{enumerate}[label=(\alph*)]
        \item Compute the controllable canonical form realization for the transfer function
        \[
            \hat{g}(s) = \frac{k}{s^n + \alpha_1s^{n-1} + \cdots + \alpha_{n-1}s + \alpha_n}
        \]
        \item For the realization in (a), compute the transfer function from the input \(u\) to the new output \(y = x_i\), where \(x_i\) is the \(i\)th element of the state \(x\).
        \item  Compute the controllable canonical form realization for the transfer function:
        \[
            \hat{g}(s) = \frac{\beta_1s^{n-1} + \beta_2s^{n-2} + \cdots + \beta_{n-1}s + \beta_n}{s^n + \alpha_1s^{n-1} + \cdots + \alpha_{n-1}s + \alpha_n}
        \]
    \end{enumerate}
\end{problem}

\begin{solution}
    \begin{enumerate}[label=(\alph*)]
        \item \(\hat{g}(s)\) is \(1 \times 1\), \(A\) has to be \(n \times n\) because of the \(\alpha\)'s. So \(C\) has to be \(1 \times n\) and \(B\) has to be \(n \times 1\).
        \[
            \hat{g}(s) = \frac{1}{d(s)} (k)
        \]
        Using the known matrices:
        \[
            \begin{aligned}
                A &= \begin{pmatrix}
                    -\alpha_1 & -\alpha_2 & -\alpha_3 & \cdots & -\alpha_n \\
                    1 & 0 & 0 & \cdots & 0 \\
                    0 & 1 & 0 & \cdots & 0 \\
                    \vdots & \cdots & \ddots & \ddots & \vdots \\
                    0 & \cdots & 0 & 1 & 0 \\
                \end{pmatrix}\\
                B &= \begin{pmatrix}
                    1 \\
                    0 \\
                    0 \\
                    \vdots \\
                    0 \\
                \end{pmatrix}\\
                C &= \begin{pmatrix}
                    0 & 0 & \cdots & 0 & k
                \end{pmatrix}\\
                D &= \begin{pmatrix}
                    0
                \end{pmatrix}
            \end{aligned}
        \]
        \item State-space system:
        \[
            \begin{aligned}
                \dot{x} &= Ax + Bu \\
                y &= Cx + Du
            \end{aligned}
        \]
        For the new output \(y = x_i\), we need to modify the output matrix \(C\) to select the \(i\)th state. The new \(C\) matrix, denoted as \(C_i\), will be:
        \[
            C_i = \begin{pmatrix}
                0 & 0 & \cdots & 0 & 1 & 0 & \cdots & 0
            \end{pmatrix}
        \]
        where the 1 is in the \(i\)th position. The new transfer function from the input \(u\) to the output \(y = x_i\) is given by:
        \[
            \hat{G}_i(s) = C_i(s - A)^{-1}B + D
        \]
        Substituting the known matrices \(A\), \(B\), \(C_i\), and \(D\), we get:
        \[
            \hat{G}_i(s) = \begin{pmatrix}
                0 & 0 & \cdots & k
            \end{pmatrix}
            \begin{pmatrix}
                s + \alpha_1 & \alpha_2 & \alpha_3 & \cdots & \alpha_n \\
                -1 & s & 0 & \cdots & 0 \\
                0 & -1 & s & \cdots & 0 \\
                \vdots & \cdots & \ddots & \ddots & \vdots \\
                0 & \cdots & 0 & -1 & s
            \end{pmatrix}^{-1}
            \begin{pmatrix}
                1 \\
                0 \\
                0 \\
                \vdots \\
                0
            \end{pmatrix}
        \]
        \item Like before but with \(\beta\)'s,
        \[
            \begin{aligned}
                A &= \begin{pmatrix}
                    -\alpha_1 & -\alpha_2 & -\alpha_3 & \cdots & -\alpha_n \\
                    1 & 0 & 0 & \cdots & 0 \\
                    0 & 1 & 0 & \cdots & 0 \\
                    \vdots & \cdots & \ddots & \ddots & \vdots \\
                    0 & \cdots & 0 & 1 & 0 \\
                \end{pmatrix}\\
                B &= \begin{pmatrix}
                    1 \\
                    0 \\
                    0 \\
                    \vdots \\
                    0 \\
                \end{pmatrix}\\
                C &= \begin{pmatrix}
                    \beta_1 & \beta_2 & \cdots & \beta_{n-1} & \beta_n
                \end{pmatrix}\\
                D &= \begin{pmatrix}
                    0
                \end{pmatrix}
            \end{aligned}
        \]
        \item Transpose and swap some stuff:
        \[
            \begin{aligned}
                A &= \begin{pmatrix}
                    -\alpha_1 & 1 & 0 & \cdots & 0 \\
                    -\alpha_2 & 0 & 1 & \cdots & 0 \\
                    -\alpha_3 & 0 & 0 & \cdots & 0 \\
                    \vdots & \vdots & \vdots & \ddots & \vdots \\
                    -\alpha_n & 0 & 0 & \cdots & 0 \\
                \end{pmatrix}\\
                B &= \begin{pmatrix}
                    \beta_1 \\
                    \beta_2 \\
                    \vdots \\
                    \beta_n
                \end{pmatrix}\\
                C &= \begin{pmatrix}
                    1 & 0 & 0 & \cdots & 0
                \end{pmatrix}\\
                D &= \begin{pmatrix}
                    0
                \end{pmatrix}
            \end{aligned}
        \]
    \end{enumerate}
\end{solution}

\newpage

\begin{problem}{3}
    Consider the following two systems:
    \[
        \begin{aligned}
            \dot{x} &= \begin{pmatrix}
                2 & 1 & 2 \\
                0 & 2 & 2 \\
                0 & 0 & 1
            \end{pmatrix}x + \begin{pmatrix}
                1 \\
                1 \\
                0
            \end{pmatrix}u, \quad y = \begin{pmatrix}
                1 & -1 & 0
            \end{pmatrix}x \\
            \dot{x} &   = \begin{pmatrix}
                2 & 1 & 1 \\
                0 & 2 & 1 \\
                0 & 0 & -1
            \end{pmatrix}x + \begin{pmatrix}
                1 \\
                1 \\
                0
            \end{pmatrix}u, \quad y = \begin{pmatrix}
                1 & -1 & 0
            \end{pmatrix}x
        \end{aligned}
    \]
    \begin{enumerate}[label=(\alph*)]
        \item Are these systems zero-state equivalent?
        \item Are they algebraically equivalent?
    \end{enumerate}
\end{problem}

\begin{solution}
    \[
        \begin{aligned}
            A &= \begin{pmatrix}
                2 & 1 & 2 \\
                0 & 2 & 2 \\
                0 & 0 & 1
            \end{pmatrix}\\
            B &= \begin{pmatrix}
                1 \\
                1 \\
                0
            \end{pmatrix}\\
            C &= \begin{pmatrix}
                1 & -1 & 0
            \end{pmatrix}\\
            D &= \begin{pmatrix}
                0
            \end{pmatrix}
        \end{aligned}
    \]
    Compute:
    \[
        \begin{aligned}
            \hat{g}_1(s) &= C(s - A)^{-1}B + D \\
            &=\begin{pmatrix}
                1 & -1 & 0
            \end{pmatrix}
            \begin{pmatrix}
                s - 2 & -1 & -2 \\
            0 & s - 2 & -2 \\
                0 & 0 & s - 1
            \end{pmatrix}^{-1}
            \begin{pmatrix}
                1 \\
                1 \\
                0
            \end{pmatrix}\\
            &= \frac{1}{(s-2)^2}
        \end{aligned}
    \]
    The other one:
    \[
        \begin{aligned}
            \hat{g}_2(s) &= C(s - A)^{-1}B + D \\
            &=\begin{pmatrix}
                1 & -1 & 0
            \end{pmatrix}
            \begin{pmatrix}
                s - 2 & -1 & -1 \\
            0 & s - 2 & -1 \\
                0 & 0 & s + 1
            \end{pmatrix}^{-1}
            \begin{pmatrix}
                1 \\
                1 \\
                0
            \end{pmatrix}\\
            &= \frac{1}{(s-2)^2}
        \end{aligned}
    \]
    They are zero-state equivalent. They are not algebraically equivalent because the eigenvalues are different.

\end{solution}

\newpage

\begin{problem}{4}
    Consider the following system:
    \[
        \dot{x} = Ax + Bu, \quad y = Cx
    \]
    with
    \[
        A = \begin{pmatrix}
            0 & 1 \\
            0 & 0
        \end{pmatrix}, \quad B = \begin{pmatrix}
            0 \\
            1
        \end{pmatrix}, \quad C = \begin{pmatrix}
            1 & 1
        \end{pmatrix}
    \]
    and the feedback state control
    \[
        u = - \begin{pmatrix}
            f_1 & f_2
        \end{pmatrix}x + v
    \]
    where \(v\) denotes a disturbance input and \(f_1\) and \(f_2\) two scalar constants to be
    specified later.
    \begin{enumerate}[label=(\alph*)]
        \item Compute the close-loop state-space model with input \(v\) and output \(y\),
        leaving your answer as a function of the constants \(f_1\) and \(f_2\).
        \item Compute the closed loop transfer function from \(v\) to \(y\), leaving your answer as a function of the constants \(f_1\) and \(f_2\).
        \item Determine values for \(f_1\) and \(f_2\) so as to obtain the following transfer function from \(v\) to \(y\):
        \[
            \frac{1}{s+1}
        \]
    \end{enumerate}
\end{problem}

\begin{solution}
    \begin{enumerate}[label=(\alph*)]
        \item
        \[
            \begin{aligned}
                \dot{x} &= Ax + Bu \\
                &= Ax + B(-Fx + v) \\
                &= (A-BF)x + Bv \\
                &= \begin{pmatrix}
                    0 & 1 \\
                    0 & 0
                \end{pmatrix}x - \begin{pmatrix}
                    0 \\
                    1
                \end{pmatrix}
                \begin{pmatrix}
                    f_1 & f_2
                \end{pmatrix}x + \begin{pmatrix}
                    0 \\
                    1
                \end{pmatrix}v \\
                &= \begin{pmatrix}
                    0 & 1 \\
                    -f_1 & -f_2
                \end{pmatrix}x + \begin{pmatrix}
                    0 \\
                    1
                \end{pmatrix}v
            \end{aligned}
        \]
        
        The output equation remains:
        \[
            y = Cx = \begin{pmatrix} 1 & 1 \end{pmatrix}x
        \]
        
        Therefore, the closed-loop state-space model is:
        \[
            \begin{aligned}
                \dot{x} &= \begin{pmatrix}
                    0 & 1 \\
                    -f_1 & -f_2
                \end{pmatrix}x + \begin{pmatrix}
                    0 \\
                    1
                \end{pmatrix}v \\
                y &= \begin{pmatrix} 1 & 1 \end{pmatrix}x
            \end{aligned}
        \]

    \item
    New transfer function:
    \[
        G(s) = C(s - (A - BF))^{-1}B
    \]
    Find \( (s - (A - BF)) \):
    \[
        s - (A - BF) = \begin{pmatrix}
            s & 0 \\
            0 & s
        \end{pmatrix} - \begin{pmatrix}
            0 & 1 \\
            -f_1 & -f_2
        \end{pmatrix} = \begin{pmatrix}
            s & -1 \\
            f_1 & s + f_2
        \end{pmatrix}
    \]
    New transfer function:
    \[
        G(s) = \begin{pmatrix}
            1 & 1
        \end{pmatrix}
        \begin{pmatrix}
            s & -1 \\
            f_1 & s + f_2
        \end{pmatrix}^{-1}
        \begin{pmatrix}
            0 \\
            1
        \end{pmatrix}
    \]

    Simplify:
    \[
        G(s) =  \frac{s + 1}{s^2 + f_2s + f_1}
    \]  
    \item 
    \[
        \begin{aligned}
            \frac{s + 1}{s^2 + f_2s + f_1} &= \frac{1}{s + 1} \\
            s^2 + 2s + 1 &= s^2 + f_2s + f_1 \\
        \end{aligned}
    \]
    So,
    \[
        f_1 = 1, \quad f_2 = 2
    \]
    \end{enumerate}
\end{solution}



\end{document}