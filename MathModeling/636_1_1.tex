\documentclass[12pt]{article}
\usepackage[utf8]{inputenc}
\usepackage{amsmath, amssymb, amsthm}
\usepackage{enumitem}
\usepackage{geometry}
\usepackage{fancyhdr}
\usepackage{pgfplots}
\usepackage{tikz}
\usepackage{float}
\usepackage{graphicx}
\DeclareMathOperator{\Tr}{Tr}
\DeclareMathOperator{\RR}{\mathbb{R}}
\DeclareMathOperator{\rng}{rng}

% Page setup
\setlength{\headheight}{15pt}
\geometry{letterpaper, margin=1in}
\setlength{\parindent}{0pt}
\setlength{\parskip}{1em}
\pagestyle{fancy}
\fancyhf{}
\fancyhead[L]{\textbf{Sebastian Griego}}  % Replace with your name
\fancyhead[C]{\textbf{Math Modeling}}  % Replace with your course name
\fancyhead[R]{\textbf{Assignment \#1}}  % Replace with your assignment number
\fancyfoot[C]{\thepage}

\newenvironment{problem}[1]{
    \textbf{Problem #1:}
}{
    \rmfamily \vspace{1em}
}

\newenvironment{solution}{
    \textbf{Solution:}
    
}{
    
    \vspace{2em}
}

\begin{document}

\title{Homework \#1}  % Replace with the homework number
\author{Sebastian Griego}  % Replace with your name

\begin{problem}{1}
    A ball is thrown directly upward from the surface of the Earth. Assuming that the maximum height reached is a monomial function of the acceleration due to gravity, the mass of the ball and the initial velocity, use dimensional analysis to approximate the expression for the maximum height reached.
\end{problem}

\begin{solution}
    Gravity: \([LT^{-2}] = [\frac{L}{T^2}]\)\\
    Velocity: \([LT^{-1}] = [\frac{L}{T}]\)\\
    Mass: \([M]\)\\
    Height: \([L]\)\\
    \[
        \begin{aligned}\relax
            [h] &= [g^a v^b m^c] \\
            [L] &= \left[\frac{L}{T^2}\right]^a \left[\frac{L}{T}\right]^b [M]^c \\
            [L] &= [L^{a+b} \cdot T^{-2a-b} \cdot M^c] \\
        \end{aligned}
    \]
    This gives
    \[
        \begin{aligned}
            a+b &= 1 \\
            -2a-b &= 0 \\
            c &= 0
        \end{aligned}
    \]
    Simplifying gives
    \[
        b =2
    \]
    Plugging this into the first equation gives
    \[
        a + 2 = 1 \implies a = -1
    \]
    Therefore,
    \[
        \begin{aligned}
            a &= -1 \\
            b &= 2 \\
            c &= 0
        \end{aligned}
    \]
    Therefore, the expression for the maximum height reached is
    \[
        h = \frac{v^2}{g}
    \]
    


    
    
\end{solution}



\end{document}

