\documentclass[12pt]{article}
\usepackage[utf8]{inputenc}
\usepackage{amsmath, amssymb, amsthm}
\usepackage{enumitem}
\usepackage{geometry}
\usepackage{fancyhdr}
\usepackage{pgfplots}
\usepackage{tikz}
\usepackage{float}
\usepackage{graphicx}
\DeclareMathOperator{\Tr}{Tr}
\DeclareMathOperator{\RR}{\mathbb{R}}
\DeclareMathOperator{\rng}{rng}

% Page setup
\setlength{\headheight}{15pt}
\geometry{letterpaper, margin=1in}
\setlength{\parindent}{0pt}
\setlength{\parskip}{1em}
\pagestyle{fancy}
\fancyhf{}
\fancyhead[L]{\textbf{Sebastian Griego}}  % Replace with your name
\fancyhead[C]{\textbf{Math Modeling}}  % Replace with your course name
\fancyhead[R]{\textbf{Assignment \#1}}  % Replace with your assignment number
\fancyfoot[C]{\thepage}

\newenvironment{problem}[1]{
\textbf{Problem #1:}
}{
\rmfamily \vspace{1em}
}

\newenvironment{solution}{
\textbf{Solution:}

}{

\vspace{2em}
}

\begin{document}

\title{Homework \#1}  % Replace with the homework number
\author{Sebastian Griego}  % Replace with your name

\begin{problem}{6}
    According to the radioactive decay law, the per capita decay rate of the amount \(A(t)\) of \(C^{14}\) is \(-\lambda\). Suppose that an archeologist excavates a bone and measures its content for \(C^{14}\) and finds it to be \(25\%\) of the carbon present in bones of a living organism. What can be said about the age of the bone? The half-life of \(C^{14}\) is \(5730\) years.
\end{problem}

\begin{solution}
    The decay rate is given by:
    \[
        \frac{dA}{dt} = -\lambda A
    \]
    The solution to this differential equation is:
    \[
        A(t) = A_0 e^{-\lambda t}
    \]
    Given that:
    \[
        \begin{aligned}
            A(5730) &= 0.5A_0 \\
            A_0 e^{-\lambda 5730} &= 0.5A_0 \\
            e^{-\lambda 5730} &= 0.5 \\
            -\lambda 5730 &= \ln(0.5) \\
            \lambda &= \frac{\ln(0.5)}{-5730}
        \end{aligned}
    \]
    Want to find \(t\) when \(A(t) = 0.25A_0\):
    \[
        \begin{aligned}
            A(t) &= 0.25A_0 \\
            A_0 e^{-\lambda t} &= 0.25A_0 \\
            e^{\frac{\ln(0.5)}{5730}t} &= 0.25 \\
            \frac{\ln(0.5)}{5730}t &= \ln(0.25) \\
            t &= \frac{5730 \ln(0.25)}{\ln(0.5)}\\
            t &= \frac{5730 \cdot 2\ln(0.5)}{\ln(0.5)}\\
            t &= 2 \cdot 5730\\
            t &= 11460
        \end{aligned}
    \]
    Therefore, the age of the bone is \(11460\) years.
\end{solution}

\end{document}

