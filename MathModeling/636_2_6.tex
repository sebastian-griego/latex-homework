\documentclass[12pt]{article}
\usepackage[utf8]{inputenc}
\usepackage{amsmath, amssymb, amsthm}
\usepackage{enumitem}
\usepackage{geometry}
\usepackage{fancyhdr}
\usepackage{pgfplots}
\usepackage{tikz}
\usepackage{float}
\usepackage{graphicx}
\DeclareMathOperator{\Tr}{Tr}
\DeclareMathOperator{\rng}{rng}
\DeclareMathOperator{\norm}{||}
\DeclareMathOperator{\NN}{\mathbb{N}}
\DeclareMathOperator{\ZZ}{\mathbb{Z}}
\DeclareMathOperator{\QQ}{\mathbb{Q}}
\DeclareMathOperator{\RR}{\mathbb{R}}
\DeclareMathOperator{\CC}{\mathbb{C}}

% Page setup
\setlength{\headheight}{15pt}
\geometry{letterpaper, margin=1in}
\setlength{\parindent}{0pt}
\setlength{\parskip}{1em}
\pagestyle{fancy}
\fancyhf{}
\fancyhead[L]{\textbf{Sebastian Griego}}  % Replace with your name
\fancyhead[C]{\textbf{Math Modeling}}  % Replace with your course name
\fancyhead[R]{\textbf{Assignment \#2}}  % Replace with your assignment number
\fancyfoot[C]{\thepage}

\newenvironment{problem}[1]{
    \textbf{Problem #1:}
}{
    \rmfamily \vspace{1em}
}

\newenvironment{solution}{
    \textbf{Solution:}
    
}{
    
    \vspace{2em}
}

\begin{document}

\title{Homework \#2}  % Replace with the homework number
\author{Sebastian Griego}  % Replace with your name

\begin{problem}{6}
    Matured insects lay eggs with per capita rate of \(r\), which survive and hatch to immature population with survival rate \(e^{-\psi x}\), where \(x\) is a number of eggs. The immature insects become matured with per capita maturation rate \(\gamma\). Assume that \(\delta\) and \(\mu\) are per capita mortality rate of immature and mature insect populations, respectively.
    \begin{enumerate}
        \item Develop a patchy model with two patches, one representing immature insects and another representing mature insects.
        \item Consider a control mechanism which results in the reduction of the egg laying rate, i.e., \(r \to r(1-\theta)r\) with the control level \(\theta\). Perform bifurcation analysis of the model to identify the level of control mechanism for extinction and for persistence of the insect population.
    \end{enumerate}
\end{problem}

\begin{solution}
    Let \( M(t) \) represent the population of mature insects, and \( I(t) \) represent the population of immature insects.
    \[
        \begin{aligned}
            \dfrac{dI}{dt} &= e^{-\psi r x} \cdot M - (\gamma + \delta) I \\
            \dfrac{dM}{dt} &= \gamma I - \mu M
        \end{aligned}
    \]
    The immature population goes to the mature population, and the mature population lays eggs that go to the immature population.

    Let \( r' = r(1 - \theta) \) replace \( r \) in the model.

    \[
    \begin{aligned}
        \dfrac{dI}{dt} &= e^{-\psi r(1 - \theta) x}  M - (\gamma + \delta) I \\
        \dfrac{dM}{dt} &= \gamma I - \mu M
    \end{aligned}
    \]

    Find the equilibrium points by setting the DEs to zero:
    \[
    \begin{aligned}
        e^{-\psi r(1 - \theta) x} M - (\gamma + \delta) I &= 0 \\
        \gamma I - \mu M &= 0
    \end{aligned}
    \]
 
    The second equation gives:

    \[
    I = \dfrac{\mu}{\gamma} M
    \]

    Substitute this into the first equation:
    \[
        e^{-\psi r(1 - \theta) x} M - (\gamma + \delta) \left( \dfrac{\mu}{\gamma} M \right ) = 0
    \]
    Simplify:
    \[
        e^{-\psi r(1 - \theta) x} M - \frac{\mu(\gamma + \delta)}{\gamma} M = 0
    \]
    
    Factor out M:
    \[
        M \left(e^{-\psi r(1 - \theta) x} - \frac{\mu(\gamma + \delta)}{\gamma}\right) = 0
    \]
    
    This equation has two solutions:
    
    1. \(M = 0\) (trivial equilibrium)
    2. \(e^{-\psi r(1 - \theta) x} = \frac{\mu(\gamma + \delta)}{\gamma}\)
    
    For the non-trivial equilibrium:
    
    \[
        \begin{aligned}
            -\psi r(1 - \theta) x &= \ln\left(\frac{\mu(\gamma + \delta)}{\gamma}\right) \\
            (1 - \theta) &= -\frac{1}{\psi rx} \ln\left(\frac{\mu(\gamma + \delta)}{\gamma}\right) \\
            -\theta &= -\frac{1}{\psi rx} \ln\left(\frac{\mu(\gamma + \delta)}{\gamma}\right) - 1 \\
            \theta &= 1 + \frac{1}{\psi rx} \ln\left(\frac{\mu(\gamma + \delta)}{\gamma}\right)
        \end{aligned}
    \]
    
    This gives us the critical value of \(\theta\) for which a non-trivial equilibrium exists. 
    
    For \(\theta < 1 + \frac{1}{\psi rx} \ln\left(\frac{\mu(\gamma + \delta)}{\gamma}\right)\), the insect population will persist.
    
    For \(\theta > 1 + \frac{1}{\psi rx} \ln\left(\frac{\mu(\gamma + \delta)}{\gamma}\right)\), the insect population will go extinct.
        
\end{solution}


\end{document}