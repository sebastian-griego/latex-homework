\documentclass[12pt]{article}
\usepackage[utf8]{inputenc}
\usepackage{amsmath, amssymb, amsthm}
\usepackage{enumitem}
\usepackage{geometry}
\usepackage{fancyhdr}
\usepackage{pgfplots}
\usepackage{tikz}
\usepackage{float}
\usepackage{graphicx}
\DeclareMathOperator{\Tr}{Tr}
\DeclareMathOperator{\rng}{rng}
\DeclareMathOperator{\norm}{||}
\DeclareMathOperator{\NN}{\mathbb{N}}
\DeclareMathOperator{\ZZ}{\mathbb{Z}}
\DeclareMathOperator{\QQ}{\mathbb{Q}}
\DeclareMathOperator{\RR}{\mathbb{R}}
\DeclareMathOperator{\CC}{\mathbb{C}}

% Page setup
\setlength{\headheight}{15pt}
\geometry{letterpaper, margin=1in}
\setlength{\parindent}{0pt}
\setlength{\parskip}{1em}
\pagestyle{fancy}
\fancyhf{}
\fancyhead[L]{\textbf{Sebastian Griego}}  % Replace with your name
\fancyhead[C]{\textbf{Math Modeling}}  % Replace with your course name
\fancyhead[R]{\textbf{Midterm 1}}  % Replace with your assignment number
\fancyfoot[C]{\thepage}

\newenvironment{problem}[1]{
    \textbf{Problem #1:}
}{
    \rmfamily \vspace{1em}
}

\newenvironment{solution}{
    \textbf{Solution:}
    
}{
    
    \vspace{2em}
}

\begin{document}

\title{Midterm 1}  % Replace with the homework number
\author{Sebastian Griego}  % Replace with your name


\begin{problem}{2}
    For a given model equation:
    \[
        \frac{dN(t)}{dt} = r_B N(t) e^{-\Psi r_B(N(t))} - B\frac{N(t)}{A + N(t)}
    \]
    perform dimensional analysis to reduce the equation to the form 
    \[
        \frac{du}{d\tau} = r u e^{-qu} - \frac{u}{1 + u}
    \]
    Use graphical techniques to perform the stability analysis and the bifurcation analysis with the parameter \( q \) fixed and the parameter \( r \) as a bifurcation parameter. Also sketch the bifurcation diagram.
\end{problem}

\begin{solution}
    Let \( N = [N]N^* \) and \( t = [t]t^* \). Then,
    \[
        \begin{aligned}
            \frac{[N]}{[t]} \frac{dN^*}{dt^*} &= r_B [N]N^* e^{-\Psi r_B [N]N^*} - B \frac{[N]N^*}{A + [N]N^*} \\
            \frac{dN^*}{dt^*} &= [t]r_B N^* e^{-\Psi r_B [N]N^*} - \frac{B [t]N^*}{A + [N]N^*}
        \end{aligned}
    \]
    Set:
    \[
        \begin{aligned}
            [N] &= A \\
            [t] &= \frac{A}{B}
        \end{aligned}
    \]
    
    Substituting these values:
    \[
        \begin{aligned}
            \frac{dN^*}{dt^*} &= \frac{r_B A}{B} N^* e^{-\Psi r_B A N^*} - \frac{B \frac{A}{B} N^*}{A + A N^*} \\
            &= \frac{r_B A}{B} N^* e^{-\Psi r_B A N^*} - \frac{N^*}{1 + N^*}
        \end{aligned}
    \]
    
    Let \( u = N^* \), \( \tau = t^* \), \( r = \frac{r_B A}{B} \), and \( q = \Psi r_B A \). Then:
    \[
        \frac{du}{d\tau} = r u e^{-qu} - \frac{u}{1 + u}
    \]
    
    \textbf{Stability and Bifurcation Analysis:}
    
    Set:
    \[
        r u e^{-qu} - \frac{u}{1 + u} = 0
    \]
    
    Trivial equilibrium is \( u = 0 \). For non-trivial equilibria:
    \[
        r e^{-qu} = \frac{1}{1 + u}
    \]
    
    Let:
    \[
        \begin{aligned}
            f(u) &= r e^{-qu} \\
            g(u) &= \frac{1}{1 + u}
        \end{aligned}
    \]
    
    The intersections of these curves give the equilibrium points.
    
    1. For \( r < 1 \), there's only one intersection at \( u = 0 \).
    
    2. For \( r > 1 \), there are two equilibria: 0 and \(u^* > 0\).
    
    Here is the picture. These are the graphs with \(u\) factored out from the equations. In it, I found that \(0\) is unstable and \(u^* > 0\) is stable.
    \begin{figure}[H]
        \centering
        \includegraphics[width=0.5\textwidth]{mid12.pdf}
        \caption{Bifurcation Diagram}
    \end{figure}
\end{solution}

\end{document}


