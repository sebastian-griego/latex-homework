\documentclass[12pt]{article}
\usepackage[utf8]{inputenc}
\usepackage{amsmath, amssymb, amsthm}
\usepackage{enumitem}
\usepackage{geometry}
\usepackage{fancyhdr}
\usepackage{pgfplots}
\usepackage{tikz}
\usepackage{float}
\usepackage{graphicx}
\DeclareMathOperator{\Tr}{Tr}
\DeclareMathOperator{\rng}{rng}
\DeclareMathOperator{\norm}{||}
\DeclareMathOperator{\NN}{\mathbb{N}}
\DeclareMathOperator{\ZZ}{\mathbb{Z}}
\DeclareMathOperator{\QQ}{\mathbb{Q}}
\DeclareMathOperator{\RR}{\mathbb{R}}
\DeclareMathOperator{\CC}{\mathbb{C}}

% Page setup
\setlength{\headheight}{15pt}
\geometry{letterpaper, margin=1in}
\setlength{\parindent}{0pt}
\setlength{\parskip}{1em}
\pagestyle{fancy}
\fancyhf{}
\fancyhead[L]{\textbf{Sebastian Griego}}  % Replace with your name
\fancyhead[C]{\textbf{ODEs}}  % Replace with your course name
\fancyhead[R]{\textbf{Assignment \#5}}  % Replace with your assignment number
\fancyfoot[C]{\thepage}

\newenvironment{problem}[1]{
    \textbf{Problem #1:}
}{
    \rmfamily \vspace{1em}
}

\newenvironment{solution}{
    \textbf{Solution:}
    
}{
    
    \vspace{2em}
}

\begin{document}

\title{Homework \#5}  % Replace with the homework number
\author{Sebastian Griego}  % Replace with your name
\maketitle

\begin{problem}{1}
    Referring to the videos on control theory, briefly describe the difference between ‘open loop’ and ‘closed loop’ control.
\end{problem}

\begin{solution}
    Open loop control is a control system where the control action is based solely on the initial conditions and the system dynamics. There is no feedback from the environment or the output of the system.

    Closed loop control is a control system where the control action is based on the output of the system. The system is affected by the output and the environment. It "continuously" changes its control action based on the output of the system. The output feeds back into the system.
\end{solution}

\newpage

\begin{problem}{2}
    From the bootcamp video, what are four reasons we want feedback in our control systems?
\end{problem}

\begin{solution}
    The four reasons we want feedback in our control systems are:
    \begin{enumerate}
        \item Uncertainty: Allows for the system to be adjusted even without a perfect model of the system.
        \item Instability: Can change the eigenvalues of the system to make it stable.
        \item Disturbances: Can detect and correct for outside disturbances.
        \item Efficient: If the system is good, then the feedback makes sure extra energy is not being used needlessly.
    \end{enumerate}
\end{solution}

\newpage

\begin{problem}{3}
    Show that \(||I|| = 1\)
\end{problem}

\begin{solution}
    Define \(||I|| = \underset{||\hat{x}|| = 1}{\max} ||I\hat{x}||\)

    Lazy proof:

    \(||I|| = \underset{||\hat{x}|| = 1}{\max} ||I\hat{x}|| = \underset{||\hat{x}|| = 1}{\max} ||\hat{x}|| = 1\)

    More formal proof:
    
    \(||I|| \leq ||I\hat{x}|| = ||\hat{x}|| = 1 \implies ||I||\leq 1\)

    Also, \(||I\hat{x}|| \leq ||I||\cdot||\hat{x}|| \implies \frac{||I\hat{x}||}{||\hat{x}||} \leq ||I||\)

    \(\implies \frac{||\hat{x}||}{||\hat{x}||} \leq ||I||\)

    \(\implies 1 \leq ||I||\)

    Therefore, \(||I|| = 1\).
\end{solution}

\newpage

\begin{problem}{4}
    If \(A\) is an invertible matrix, show that \(||A|| \left|\left|A^{-1}\right|\right| \geq 1\).
\end{problem}

\begin{solution}
    Using \(||AB|| \leq ||A|| \left|\left|B\right|\right|\) and the previous problem:

    \(||A|| \left|\left|A^{-1}\right|\right| \geq ||A A^{-1}|| = ||I|| = 1\)
\end{solution}

\newpage

\begin{problem}{5}
    Prove that \(||A^m|| \leq ||A||^m\) for \(m \in \NN\).
\end{problem}

\begin{solution}
    Induction:

    Base case: \(m = 1\)

    \(||A^1|| = ||A|| \leq ||A||^1\)

    Inductive step: Assume \(||A^m|| \leq ||A||^m\) for some \(m \in \NN\).

    Then, \(||A^{m+1}|| = ||A^m A|| \leq ||A^m|| ||A|| \leq ||A||^m ||A|| = ||A||^{m+1}\)

    Therefore, by induction, \(||A^m|| \leq ||A||^m\) for all \(m \in \NN\).
\end{solution}

\newpage

\begin{problem}{6}
    Prove that if \(||A|| < 1\) then
    \[
    (I - A)^{-1} = \sum_{j=0}^{\infty} A^j
    \]
\end{problem}

\begin{solution}
    Suppose \(||A|| < 1\) and define
    \[
    B_N = \sum_{j=0}^{N} A^j
    \]
    Then,
    \[
        \begin{aligned}
            B_N (I - A) &= B_N - B_N A \\
            &= \left(\sum_{j=0}^{N} A^j\right) - \left(\sum_{j=0}^{N} A^j\right) A \\
            &= \sum_{j=0}^{N} A^j - \sum_{j=0}^{N} A^{j+1} \\
            &= \sum_{j=0}^{N} A^j - \sum_{j=1}^{N+1} A^j\\
            &= A^0 - A^{N+1} \\
            &= I - A^{N+1}
        \end{aligned}
    \]

    Therefore,
    \[
        \lim_{N \to \infty} ||B_N (I - A) - I|| = \lim_{N \to \infty} ||-A^{N+1}|| = \lim_{N \to \infty} ||A^{N+1}||
    \]
    Using the assumption that \(||A|| < 1\) and the fact that \(||A^j|| \leq ||A||^j\), we get

    \[
        \lim_{N \to \infty} ||A||^{N+1} = 0 \implies \lim_{N \to \infty} ||A^{N+1}|| = 0 \implies \lim_{N \to \infty} ||B_N (I - A) - I|| = 0
    \]

    Therefore,
    \[
        \lim_{N \to \infty} B_N = (I - A)^{-1}
    \]

    Therefore,
    \[
        (I - A)^{-1} = \sum_{j=0}^{\infty} A^j
    \]
\end{solution}


\end{document}