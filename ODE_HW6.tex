\documentclass[12pt]{article}
\usepackage[utf8]{inputenc}
\usepackage{amsmath, amssymb, amsthm}
\usepackage{enumitem}
\usepackage{geometry}
\usepackage{fancyhdr}
\usepackage{pgfplots}
\usepackage{tikz}
\usepackage{float}
\usepackage{graphicx}
\DeclareMathOperator{\Tr}{Tr}
\DeclareMathOperator{\rng}{rng}
\DeclareMathOperator{\norm}{||}
\DeclareMathOperator{\NN}{\mathbb{N}}
\DeclareMathOperator{\ZZ}{\mathbb{Z}}
\DeclareMathOperator{\QQ}{\mathbb{Q}}
\DeclareMathOperator{\RR}{\mathbb{R}}
\DeclareMathOperator{\CC}{\mathbb{C}}

% Page setup
\setlength{\headheight}{15pt}
\geometry{letterpaper, margin=1in}
\setlength{\parindent}{0pt}
\setlength{\parskip}{1em}
\pagestyle{fancy}
\fancyhf{}
\fancyhead[L]{\textbf{Sebastian Griego}}  % Replace with your name
\fancyhead[C]{\textbf{ODEs}}  % Replace with your course name
\fancyhead[R]{\textbf{Assignment \#6}}  % Replace with your assignment number
\fancyfoot[C]{\thepage}

\newenvironment{problem}[1]{
    \textbf{Problem #1:}
}{
    \rmfamily \vspace{1em}
}

\newenvironment{solution}{
    \textbf{Solution:}
    
}{
    
    \vspace{2em}
}

\begin{document}

\title{Homework \#6}  % Replace with the homework number
\author{Sebastian Griego}  % Replace with your name
\maketitle

\begin{problem}{1}
    Consider the system
    \[
        \begin{pmatrix}
            \dot{x}_1(t) \\
            \dot{x}_2(t)
        \end{pmatrix} = \begin{pmatrix}
            0 & t \\
            0 & 2
        \end{pmatrix}\begin{pmatrix}
            x_1 \\
            x_2
        \end{pmatrix} + \begin{pmatrix}
            0 \\
            t
        \end{pmatrix} u, \quad y = \begin{pmatrix}
            1 & 0 \end{pmatrix} x, \quad x \in \RR^2, \quad u,y \in \RR
    \]
    \begin{enumerate}[label=\alph*)]
        \item Compute its state transition matrix.
        \item Compute the system’s output to the constant input \(u(t) = 1, \forall t \geq 0\), for an arbitrary initial condition \(x(0) = \begin{pmatrix}
            x_1(0) \\
            x_2(0)
        \end{pmatrix}\).
    \end{enumerate}
\end{problem}

\begin{solution}
    \begin{enumerate}[label=\alph*)]
        \item Want to find \(\Phi(t,s)\).
        \[
            \vec{x}(t) = \Phi(t,t_0)\vec{x}(0) + \int_{t_0}^t \Phi(t,s)\vec{B}(s)u(s)ds
        \]
        \[
            \begin{aligned}
                \dot{x_1}(t) &= tx_2(t) \\
                \dot{x_2}(t) &= 2x_2(t)
            \end{aligned}
        \]
        So,
        \[
            x_2(t) = x_{2,0}e^{2t-t_0}
        \]
        Now,
        \[
            \begin{aligned}
                x_1(t) &= x_{1,0} + x_{2,0} \int_{t_0}^t (s-t_0+t_0) e^{2(s-t_0)} ds \\
                &=x_{1,0} + x_{2,0} \int_{0}^{t-t_0} (u+t_0) e^{2u} du \\
                &= x_{1,0} + x_{2,0} \left(\frac{u+t_0}{2}e^{2u} \right)_{0}^{t-t_0} - x_{2,0} \int_{0}^{t-t_0} \frac{1}{2}e^{2u} du \\
                &= x_{1,0} + x_{2,0} \left(\frac{t-t_0+t_0}{2}e^{2(t-t_0)} - \frac{t_0}{2} \right) - x_{2,0} \left( \frac{1}{4}e^{2(t-t_0)} - \frac{1}{4} \right) \\
                &= x_{1,0} + x_{2,0} \left(\frac{t}{2}e^{2(t-t_0)} - \frac{t_0}{2} - \frac{1}{4}e^{2(t-t_0)} + \frac{1}{4} \right) \\
            \end{aligned}
        \]
        So,
        \[
            \begin{pmatrix}
                x_1(t) \\
                x_2(t)
            \end{pmatrix} = \begin{pmatrix}
                1 & \Phi(t,t_0) \\
                0 & e^{2(t-t_0)}
            \end{pmatrix} \begin{pmatrix}
                x_{1,0} \\
                x_{2,0}
            \end{pmatrix}
        \]
        So,
        \[
            \Phi(t,s) = \begin{pmatrix}
                1 & \frac{t}{2}e^{2(t-s)} - \frac{s}{2} - \frac{1}{4}e^{2(t-s)} + \frac{1}{4} \\
                0 & e^{2(t-s)}
            \end{pmatrix}
        \]
        \item part b
        
        The soltion is
        \[
            \vec{x}(t) = \Phi(t,t_0)\vec{x}(0) + \int_{t_0}^t \Phi(t,s)\vec{B}(s)u(s)ds
        \]
        Here, \(u(s) = 1\), so
        \[
            \vec{x}(t) = \Phi(t,t_0)\vec{x}(0) + \int_{t_0}^t \Phi(t,s)\vec{B}(s)ds
        \]
        It is given that \(B(s) = \begin{pmatrix}
            0 \\
            s
        \end{pmatrix}\), so
        \[
            \vec{x}(t) = \Phi(t,t_0)\vec{x}(0) + \int_{t_0}^t \Phi(t,s)\begin{pmatrix}
            0 \\
            s
        \end{pmatrix}ds
        \]
        So,
        \[
            \begin{pmatrix}
                x_1(t) \\
                x_2(t)
            \end{pmatrix} = \begin{pmatrix}
                1 & \frac{t}{2}e^{2(t-t_0)} - \frac{s}{2} - \frac{1}{4}e^{2(t-t_0)} + \frac{1}{4} \\
                0 & e^{2(t-t_0)}
            \end{pmatrix} \begin{pmatrix}
                x_{1,0} \\
                x_{2,0}
            \end{pmatrix} + \int_{t_0}^t \begin{pmatrix}
                1 & \frac{s}{2}e^{2(s-t_0)} - \frac{s}{2} - \frac{1}{4}e^{2(s-t_0)} + \frac{1}{4} \\
                0 & e^{2(s-t_0)}
            \end{pmatrix} \begin{pmatrix}
                0 \\
                s
            \end{pmatrix}ds
            \]
            Wolfram Alpha said this was too many characters when I put it in. So, here is ChatGPT's answer (probably wrong):
            \[
                \begin{pmatrix}
                    x_1(t) \\
                    x_2(t)
                \end{pmatrix} = 
                \begin{pmatrix}
                    x_{1,0} + \left( \frac{t}{2} - \frac{1}{4} \right) e^{2(t-t_0)} - \frac{t_0}{2} + \frac{1}{4} \\
                    \left( x_{2,0} + \frac{t_0}{2} + \frac{1}{4} \right) e^{2(t-t_0)} - \frac{t}{2} - \frac{1}{4}
                \end{pmatrix}
            \]


    \end{enumerate}

\end{solution}

\newpage

\begin{problem}{2}
    Consider the homogeneous linear time-varying system
    \[
        \dot{\vec{x}}(t) = A(t)\vec{x}(t), \quad \vec{x}(t_0) = \vec{x}_0
    \]
    with state transition matrix \(\Phi(t,\tau)\). Consider also the non-homogeneous system
    \[
        \dot{\vec{z}}(t) = A(t)\vec{z}(t) + \vec{x}(t), \quad \vec{z}(t_0) = \vec{z}_0
    \]
    \begin{enumerate}[label=\alph*)]
        \item Compute \(\vec{x}(t)\) and \(\vec{z}(t)\) as a function of \(\vec{x}_0, \vec{z}_0,\) and \(\Phi(t,\tau)\). No integrals should appear in your answer.
        \item For a given time \(T > 0\), how should \(x_0\) and \(z_0\) be related to have \(z(T) = 0\)?
    \end{enumerate}
\end{problem}


\begin{solution}
    \begin{enumerate}[label=\alph*)]
        \item Start with \(\vec{x}(t)\)
    
        This has a straightforward solution:
        \[
            \vec{x}(t) = \Phi(t, t_0) \vec{x}_0
        \]
    
        Now for \(\vec{z}(t)\):
        \[
            \dot{\vec{z}}(t) = A(t)\vec{z}(t) + \vec{x}(t), \quad \vec{z}(t_0) = \vec{z}_0
        \]
        Start with the solution with an integral and simplify:
        \[
            \begin{aligned}
                \vec{z}(t) &= \Phi(t, t_0) \vec{z}_0 + \int_{t_0}^{t} \Phi(t, s) \vec{x}(s) \, ds \\
                &= \Phi(t, t_0) \vec{z}_0 + \int_{t_0}^{t} \Phi(t, s) \Phi(s, t_0) \vec{x}_0 \, ds\\
                &= \Phi(t, t_0) \vec{z}_0 + \int_{t_0}^{t} \Phi(t, t_0) \vec{x}_0 \, ds\\
                &= \Phi(t, t_0) \vec{z}_0 + \Phi(t, t_0) \vec{x}_0 \int_{t_0}^{t} ds\\
                &= \Phi(t, t_0) \vec{z}_0 + \Phi(t, t_0) \vec{x}_0 (t - t_0)
            \end{aligned}
        \]
        \item To achieve \(\vec{z}(T) = 0\), set the solution at \(t = T\) to zero:
        \[
            \begin{aligned}
                \vec{z}(T) &= \Phi(T, t_0) \vec{z}_0 + \Phi(T, t_0) \vec{x}_0 (T - t_0)\\
                &= \Phi(T, t_0) (\vec{z}_0 + \vec{x}_0 (T - t_0)) = 0
            \end{aligned}
        \]
        Assuming \(\Phi(T, t_0)\) is invertible (I think it has to be), we can solve for \(\vec{z}_0\):
        \[
            \vec{z}_0 + \vec{x}_0 (T - t_0) = 0 \quad \Rightarrow \quad \vec{z}_0 = -\vec{x}_0 (T - t_0)
        \]
        Therefore, the initial conditions must satisfy
        \[
            \vec{z}_0 = - (T - t_0) \vec{x}_0\,.
        \]
    \end{enumerate}
\end{solution}

\newpage

\begin{problem}{3}
    Compute \(A^n\) and \(e^{An}\) for the following matrices:
    \[
        A_1 = \begin{pmatrix}
            1 & 1 & 0 \\
            0 & 1 & 0 \\
            0 & 0 & 1
        \end{pmatrix}, \quad A_2 = \begin{pmatrix}
            1 & 1 & 0 \\
            0 & 0 & 1 \\
            0 & 0 & 1
        \end{pmatrix}, \quad A_3 = \begin{pmatrix}
            2 & 0 & 0 & 0 \\
            2 & 2 & 0 & 0 \\
            0 & 0 & 3 & 3 \\
            0 & 0 & 0 & 3
        \end{pmatrix}
    \]
\end{problem}

\begin{solution}
    \[
        \begin{aligned}
            A_1 & = I + N\\
            &= \begin{pmatrix}
                1 & 0 & 0 \\
                0 & 1 & 0 \\
                0 & 0 & 1
            \end{pmatrix} + \begin{pmatrix}
                0 & 1 & 0 \\
                0 & 0 & 0 \\
                0 & 0 & 0
            \end{pmatrix}
        \end{aligned}
    \]
    \(NI = IN\) because identity.
    \[
        \begin{aligned}
            N^2 = \begin{pmatrix}
                \begin{pmatrix}
                    0 & 1 \\
                    0 & 0
                \end{pmatrix}^2 & 0 \\
                0 & (0)^2
            \end{pmatrix} = \begin{pmatrix}
                0 & 0 \\
                0 & 0
            \end{pmatrix}
        \end{aligned}
    \]
    Binomial theorem:
    \[
        \begin{aligned}
            (I + N)^n &= \sum_{k=0}^n \binom{n}{k} I^{n-k}N^k\\
            &= \sum_{k=0}^1 \binom{n}{k} I^{n-k}N^k\\
            &= I + nN\\
            &= \begin{pmatrix}
                1 & n & 0 \\
                0 & 1 & 0 \\
                0 & 0 & 1
            \end{pmatrix}
        \end{aligned}
    \]
    Now \(A_2\):
    \[
        \begin{aligned}
            A_2 &= D + N\\
            &= \begin{pmatrix}
                1 & 0 & 0 \\
                0 & 0 & 0 \\
                0 & 0 & 1
            \end{pmatrix} + \begin{pmatrix}
                0 & 1 & 0 \\
                0 & 0 & 1 \\
                0 & 0 & 0
            \end{pmatrix}
        \end{aligned}
    \]
    where
    \[
        N^2 = \begin{pmatrix}
            0 & 0 & 1 \\
            0 & 0 & 0 \\
            0 & 0 & 0
        \end{pmatrix}
    \]
    \[
        N^3 = 0
    \]
    \[
        D^n = D
    \]
    \[
        DN + ND = N
    \]
    Look at the pattern:
    \[
        \begin{aligned}
            (D+N)^2 &= D^2 + 2DN + N^2 = D + N + N^2\\
            (D+N)^3 &= D + N + N^2 + DN^2 = D + N + 2N^2
        \end{aligned}
    \]
    Induction:
    \[
        (D+N)^1 = D + N
    \]
    Assume:
    \[
        (D+N)^n = D + N + (n-1)N^2
    \]
    Then:
    \[
        \begin{aligned}
            (D+N)^{n+1} &= (D+N)^n + (D+N)N^2\\
            &= D + N + (n-1)N^2 + DN^2\\
            &= D + N + nN^2
        \end{aligned}
    \]
    Now \(A_3\):
    \[
        \begin{aligned}
            A_3^n &= \begin{pmatrix}
                2 & 0 & 0 & 0 \\
                2 & 2 & 0 & 0 \\
                0 & 0 & 3 & 3 \\
                0 & 0 & 0 & 3
            \end{pmatrix}^n\\
            &= \begin{pmatrix}
                \begin{pmatrix}
                    2 & 0 \\
                    2 & 2
                \end{pmatrix}^n & 0 \\
                0 & \begin{pmatrix}
                    3 & 3 \\
                    0 & 3
                \end{pmatrix}^n
            \end{pmatrix}\\
            &= \begin{pmatrix}
                2^n \begin{pmatrix}
                    1 & 0 \\
                    1 & 1
                \end{pmatrix}^n & 0 \\
                0 & 3^n \begin{pmatrix}
                    1 & 1 \\
                    0 & 1
                \end{pmatrix}^n
            \end{pmatrix}\\
            &= \begin{pmatrix}
                2^n \begin{pmatrix}
                    1 & 0 \\
                    n & 1
                \end{pmatrix} & 0 \\
                0 & 3^n \begin{pmatrix}
                    1 & n \\
                    0 & 1
                \end{pmatrix}
            \end{pmatrix}
        \end{aligned}
    \]
    Now \(e^{A_1 t}\):
    \[
        \begin{aligned}
            e^{A_1 t} &= \sum_{j=0}^\infty \frac{(tA_1)^j}{j!}\\
            &= \sum_{j=0}^\infty \frac{t^j(I + N)^j}{j!}\\
            &= \sum_{j=0}^\infty \frac{t^j}{j!} \begin{pmatrix}
                1 & j & 0 \\
                0 & 1 & 0 \\
                0 & 0 & 1
            \end{pmatrix}\\
            &= e^t \sum_{j=0}^\infty \begin{pmatrix}
                1 & j & 0 \\
                0 & 1 & 0 \\
                0 & 0 & 1
            \end{pmatrix}\\
        \end{aligned}
    \]
    
    For \(e^{A_2t}\):
    \[
        \begin{aligned}
            e^{A_2t} &= \sum_{n=0}^\infty \frac{(tA_2)^n}{n!} \\
            &= \sum_{j=0}^\infty \frac{t^j(D + N)^j}{j!}\\
            &= \sum_{j=0}^\infty \frac{t^j(D + N + (j-1)N^2)}{j!}\\
        \end{aligned}
    \]
    For \(e^{A_3 t}\):
    \[
        \begin{aligned}
            e^{A_3 t} &= \sum_{j=0}^\infty \frac{(tA_3)^j}{j!}\\
            &= \sum_{j=0}^\infty \frac{t^j}{j!} \begin{pmatrix}
                2^j & 0 & 0 & 0 \\
                j2^j & 2^j & 0 & 0 \\
                0 & 0 & 3^j & j3^j \\
                0 & 0 & 0 & 3^j
            \end{pmatrix}\\
            &= e^t \sum_{j=0}^\infty \begin{pmatrix}
                2^j & 0 & 0 & 0 \\
                j2^j & 2^j & 0 & 0 \\
                0 & 0 & 3^j & j3^j \\
                0 & 0 & 0 & 3^j
            \end{pmatrix}\\
        \end{aligned}
    \]
\end{solution}

\newpage

\begin{problem}{4}
    Consider an upper triangular matrix \(A\).
    \begin{enumerate}[label=\alph*)]
        \item Show that \(e^{At}\) is also upper triangular.
        \item Relate the diagonal elements of \(A\) with those of \(e^{At}\).
    \end{enumerate}
\end{problem}

\begin{solution}
    \begin{enumerate}[label=\alph*)]
        \item Suppose \(A = \begin{pmatrix}
            a_{11} & a_{12} & \cdots & a_{1n} \\
            0 & a_{22} & \cdots & a_{2n} \\
            \vdots & \vdots & \ddots & \vdots \\
            0 & 0 & \cdots & a_{nn}
        \end{pmatrix}\).
        \[
            A = D + N
        \]
        where \(D\) is diagonal and \(N\) is strictly upper triangular.

        Put \(e^{At}\) in same form:
        \[
            e^{At} = e^{Dt}e^{Nt} = \sum_{j=0}^\infty \frac{(Dt)^j}{j!} \left(\sum_{k=0}^\infty \frac{(Nt)^k}{k!}\right)
        \]
        \(D\) remains diagonal always, and \(N\) stays upper triangular. So, \(e^{At}\) is upper triangular.

        \item Relate the diagonal elements of \(A\) with those of \(e^{At}\).


The diagonal elements of \(e^{At}\) are given by:
\[
e^{Dt} = \sum_{j=0}^{\infty} \frac{(a_{ii} t)^j}{j!} = e^{a_{ii} t}
\]
Each diagonal element of \(e^{At}\) is the exponential of the corresponding diagonal element of \(A\) multiplied by \(t\). 

\[
A = \begin{pmatrix}
a_{11} & * & \cdots & * \\
0 & a_{22} & \cdots & * \\
\vdots & \vdots & \ddots & \vdots \\
0 & 0 & \cdots & a_{nn}
\end{pmatrix},
\]
\[
e^{At} = \begin{pmatrix}
e^{a_{11} t} & * & \cdots & * \\
0 & e^{a_{22} t} & \cdots & * \\
\vdots & \vdots & \ddots & \vdots \\
0 & 0 & \cdots & e^{a_{nn} t}
\end{pmatrix}.
\]

    \end{enumerate}
    
\end{solution}

\newpage

\begin{problem}{5}
    For \(n \times n\) matrices \(A, B\), show that if \(AB = BA\), then \(e^{A+B} = e^A e^B\).

    Using this result, show then that \((e^A)^{-1} = e^{-A}\).
\end{problem}

\begin{solution}
    \[
        \begin{aligned}
            e^{A+B} &= \sum_{j=0}^\infty \frac{(A+B)^j}{j!}\\
            &= \sum_{j=0}^\infty \sum_{\ell=0}^j \frac{1}{j!} \binom{j}{\ell} A^{\ell}B^{j-\ell}\\
            &= \sum_{\ell=0}^\infty \sum_{j=\ell}^\infty \frac{1}{j!} \binom{j}{\ell} A^{\ell}B^{j-\ell}\\
            &= \sum_{\ell=0}^\infty \sum_{j=\ell}^\infty \frac{j!}{j! \ell! (j-\ell!)}A^{\ell}B^{j-\ell}\\
            &= \sum_{\ell=0}^\infty \sum_{j=\ell}^\infty \frac{1}{ \ell! (j-\ell!)}A^{\ell}B^{j-\ell}\\
            &= \sum_{\ell=0}^\infty \sum_{j-\ell = 0}^\infty \frac{1}{ \ell! (j-\ell!)}A^{\ell}B^{j-\ell}\\
            &= \left( \sum_{\ell=0}^\infty \frac{A^{\ell}}{\ell!} \right) \left( \sum_{j-\ell = 0}^\infty \frac{B^{j-\ell}}{(j-\ell)!} \right)\\
            &= e^A e^B
        \end{aligned}
    \]
    Showing \((e^A)^{-1} = e^{-A}\):
    \[
        e^{A} e^{-A} = e^{A + (-A)} = e^{A - A} = e^{0} = I
    \]
    Therefore, \(e^{-A}\) is the inverse of \(e^A\):
    \[
        e^{-A} = (e^A)^{-1}
    \]

\end{solution}



\end{document}
