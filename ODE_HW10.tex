\documentclass[12pt]{article}
\usepackage[utf8]{inputenc}
\usepackage{amsmath, amssymb, amsthm}
\usepackage{enumitem}
\usepackage{geometry}
\usepackage{fancyhdr}
\usepackage{pgfplots}
\usepackage{tikz}
\usepackage{float}
\usepackage{graphicx}
\DeclareMathOperator{\Tr}{Tr}
\DeclareMathOperator{\rng}{rng}
\DeclareMathOperator{\NN}{\mathbb{N}}
\DeclareMathOperator{\ZZ}{\mathbb{Z}}
\DeclareMathOperator{\QQ}{\mathbb{Q}}
\DeclareMathOperator{\RR}{\mathbb{R}}
\DeclareMathOperator{\CC}{\mathbb{C}}
\newcommand{\norm}[1]{\left\lVert #1 \right\rVert}


% Page setup
\setlength{\headheight}{15pt}
\geometry{letterpaper, margin=1in}
\setlength{\parindent}{0pt}
\setlength{\parskip}{1em}
\pagestyle{fancy}
\fancyhf{}
\fancyhead[L]{\textbf{Sebastian Griego}}  % Replace with your name
\fancyhead[C]{\textbf{ODEs}}  % Replace with your course name
\fancyhead[R]{\textbf{Assignment \#10}}  % Replace with your assignment number
\fancyfoot[C]{\thepage}

\newenvironment{problem}[1]{
    \textbf{Problem #1:}
}{
    \rmfamily \vspace{1em}
}

\newenvironment{solution}{
    \textbf{Solution:}
    
}{
    
    \vspace{2em}
}

\begin{document}

\title{Homework \#10}  % Replace with the homework number
\author{Sebastian Griego}  % Replace with your name
\maketitle

\begin{problem}{1}
Consider the continuous-time LTI system
\[
    \dot{x} = Ax + Bu \quad x \in \RR^n, u \in \RR^k \tag{AB-CLTI}
\]
Prove the following two statements:
\begin{enumerate}[label=(\alph*)]
    \item The controllable subspace \(\mathcal{C}\) of the system (AB-CLTI) is \(A\)-invariant.
    \item The controllable subspace \(\mathcal{C}\) of the system (AB-CLTI) contains \(Im (B)\).
\end{enumerate}
\end{problem}

\begin{solution}
    \begin{enumerate}[label=(\alph*)]
        \item 

        Let \(v \in \mathcal{C}\).

        \[
            \mathcal{C} = Im(\tilde{C}) \implies \exists x \in \tilde{V} : v = \tilde{C}x.
        \]

        Then 
        \[
            \begin{aligned}
                Av &= A\tilde{C}x \\
                &= A \begin{pmatrix}
                    B & AB & A^2B & \cdots & A^{n-1}B
                \end{pmatrix} x \\
                &= \begin{pmatrix}
                    AB & A^2B & \cdots & A^nB
                \end{pmatrix} x.
            \end{aligned}
        \]
        By Cayley-Hamilton, \(A^n = \sum_{i=0}^{n-1} a_i A^i\), so
        \[
            Av = \begin{pmatrix}
                AB & A^2B & \cdots & A^{n-1}B & \sum_{i=0}^{n-1} a_i A^{i}B
            \end{pmatrix} x.
        \]

        Therefore \(Av \in \mathcal{C}\).

        \item Let \(v \in Im(B)\). Then \(\exists x \in \tilde{V} : v = Bx\).

        Remember that \(\mathcal{C} = Im(\tilde{C})\) and

        \[
            \tilde{C} = \begin{pmatrix} B & AB & A^2B & \cdots & A^{n-1}B \end{pmatrix}
        \]

        Then \(v = Bx = \tilde{C}\begin{pmatrix} x \\ 0 \\ 0 \\ \vdots \\ 0 \end{pmatrix}\).

        Therefore, \(v \in Im(\tilde{C}) = \mathcal{C}\).

        Therefore, \(Im(B) \subseteq \mathcal{C}\).
    \end{enumerate}
\end{solution}

\newpage

\begin{problem}{2}
Consider a system in controllable canonical form
\[
    A = \begin{bmatrix}
        -\alpha_1 I_{k\times k} & -\alpha_2 I_{k\times k} & \cdots & -\alpha_{n-1} I_{k\times k} & -\alpha_n I_{k\times k} \\
        I_{k\times k} & 0_{k\times k} & \cdots & 0_{k\times k} & 0_{k\times k} \\
        0_{k\times k} & I_{k\times k} & \cdots & 0_{k\times k} & 0_{k\times k} \\
        \vdots & \vdots & \ddots & \vdots & \vdots \\
        0_{k\times k} & 0_{k\times k} & \cdots & I_{k\times k} & 0_{k\times k}
    \end{bmatrix}_{nk\times nk},
\]
\[
    B = \begin{bmatrix}
        I_{k\times k} \\
        0_{k\times k} \\
        \vdots \\
        0_{k\times k} \\
        0_{k\times k}
    \end{bmatrix}_{nk\times k},
\]
\[
    C = \begin{bmatrix}
        N_1 & N_2 & \cdots & N_{n-1} & N_n
    \end{bmatrix}_{m\times nk}.
\]
Show that such a system is always controllable.
\end{problem}

\begin{solution}
    \[
        \begin{aligned}
            \tilde{C} &= \begin{pmatrix}
                B & AB & A^2B & \cdots & A^{n-1}B
            \end{pmatrix} \\
            &= \begin{pmatrix}
                I_{k\times k} & -\alpha_1 I_{k\times k} & (\alpha_1^2 - \alpha_2)I_{k\times k} & \cdots & * \\
                0_{k\times k} & I_{k\times k} & -\alpha_1 I_{k\times k} & \cdots & * \\
                0_{k\times k} & 0_{k\times k} & I_{k\times k} & \cdots & * \\
                \vdots & \vdots & \vdots & \ddots & \vdots \\
                0_{k\times k} & 0_{k\times k} & 0_{k\times k} & \cdots & I_{k\times k}
            \end{pmatrix} \\
        \end{aligned}
    \]
    This matrix has full rank since it is triangular with identity matrices on the diagonal. Therefore, the system is controllable.
\end{solution}

\newpage

\begin{problem}{3}
Consider the SISO LTI system in controllable canonical form
\[
    \dot{x} = Ax + Bu, \quad x \in \RR^n, u \in \RR^1,
\]
where
\[
    A = \begin{bmatrix}
        -\alpha_1 & -\alpha_2 & \cdots & -\alpha_{n-1} & -\alpha_n \\
        1 & 0 & \cdots & 0 & 0 \\
        0 & 1 & \cdots & 0 & 0 \\
        \vdots & \vdots & \ddots & \vdots & \vdots \\
        0 & 0 & \cdots & 1 & 0
    \end{bmatrix}_{n\times n},
    \quad
    B = \begin{bmatrix}
        1 \\
        0 \\
        \vdots \\
        0 \\
        0
    \end{bmatrix}_{n\times 1}.
\]

\begin{enumerate}[label=(\alph*)]   
    \item Compute the characteristic polynomial of the closed-loop system for
    \[
        u = -Kx, \quad K := \begin{bmatrix} k_1 & k_2 & \cdots & k_n \end{bmatrix}.
    \]
    \textit{Hint: Compute the determinant of $(sI - A + BK)$ by doing a Laplace expansion along the first line of this matrix.}

    \item Suppose you are given $n$ complex numbers $\lambda_1, \lambda_2, \ldots, \lambda_n$ as desired locations for the closed-loop eigenvalues. Which characteristic polynomial for the closed-loop system would lead to these eigenvalues?

    \item Based on the answers to parts (a) and (b), propose a procedure to select $K$ that would result in the desired values for the closed-loop eigenvalues.
    \item Suppose that
    \[
        A = \begin{bmatrix}
            1 & 2 & 3 \\
            1 & 0 & 0 \\
            0 & 1 & 0
        \end{bmatrix}, \quad
        B = \begin{bmatrix}
            1 \\
            0 \\
            0
        \end{bmatrix}.
    \]
    Find a matrix $K$ for which the closed-loop eigenvalues are $\{-1, -1, -2\}$.
\end{enumerate}
\end{problem}

\begin{solution}
    \begin{enumerate}[label=(\alph*)]
        \item Find:
        \[
            \begin{aligned}
                \det(sI - A + BK) &= \det\begin{pmatrix}
                    s + \hat{\alpha}_1 & \hat{\alpha}_2 & \cdots & \hat{\alpha}_{n-1} & \hat{\alpha}_n\\
                    -1 & s & 0 & \cdots & 0 \\
                    0 & -1 & s & \ddots & \vdots\\
                    \vdots & \ddots & \ddots & \ddots & 0\\
                    0 & \cdots & 0 & -1 & s
                \end{pmatrix}\\
                &= s^n + \hat{\alpha}_1 s^{n-1} + \hat{\alpha}_2 s^{n-2} + \cdots + \hat{\alpha}_n\\
                &= p(s)
            \end{aligned}
        \]
        Matt and I proved this with induction. Here is a sideways picture:
        \begin{figure}[H]
            \centering
            \includegraphics[width=0.5\textwidth]{induc.jpg}
        \end{figure}
        \item \(f(s) = (s - \lambda_1)(s - \lambda_2) \cdots (s - \lambda_n)\)
        \item Match coefficients of \(f(s)\) and \(p(s)\).
        \item Follow the process of the previous parts.
        
        \[
            A = \begin{bmatrix}
                1 & 2 & 3 \\
                1 & 0 & 0 \\
                0 & 1 & 0
            \end{bmatrix}, \quad
            B = \begin{bmatrix}
                1 \\
                0 \\
                0
            \end{bmatrix}.
        \]
        Choose \(K\) such that \(\lambda = \{-1, -1, -2\}\).

        Want \(\det(sI - A + BK) = (s + 1)^2(s + 2) = s^3 + 4s^2 + 5s + 2\).

        Compute \(\det(sI - A + BK) = s^3 + (-1 + k_1)s^2 + (-2 + k_2)s + (-3 + k_3)\).

        Set coefficients equal to get \(k_1 = 5\), \(k_2 = 7\), \(k_3 = 5\).
        
        Therefore,
        \[
            K = \begin{bmatrix}
                5 & 7 & 5
            \end{bmatrix}.
        \]
    \end{enumerate}
\end{solution}

\newpage

\begin{problem}{4}
The equations of motion of a satellite linearized around a steady-state solution, are given by
\[
    \dot{x} = Ax + Bu, \quad A := \begin{bmatrix}
        0 & 1 & 0 & 0\\
        3\omega^2 & 0 & 0 & 2\omega\\
        0 & 0 & 0 & 1\\
        0 & -2\omega & 0 & 1
    \end{bmatrix}, \quad B := \begin{bmatrix}
        0 & 0\\
        1 & 0\\
        0 & 0\\
        0 & 1
    \end{bmatrix},
\]
where the state vector \(x := \begin{bmatrix} x_1 & x_2 & x_3 & x_4 \end{bmatrix}^T\) includes the perturbation \(x_1\) in the orbital radius, the perturbation \(x_2\) in the radial velocity, the perturbation \(x_3\) in the angle, and the perturbation \(x_4\) in the angular velocity; and the input vector \(u := \begin{bmatrix} u_1 & u_2 \end{bmatrix}^T\) includes the radial thruster \(u_1\) and a tangential thruster \(u_2\).

\begin{enumerate}[label=(\alph*)]
    \item Show that the system is controllable from the input vector \(u\).
    \item Can the system still be controlled if the radial thruster does not fire? What if it is the tangential thruster that fails?
\end{enumerate}
\end{problem}

\begin{solution}
    \begin{enumerate}[label=(\alph*)]
        \item Will show \(\tilde{C}\) has full rank.
        
        \[
            \begin{aligned}
                \tilde{C} &= \begin{pmatrix}
                    B & AB & A^2B & A^3B
                \end{pmatrix}\\
                &= \begin{pmatrix}
                    0 & 0 & 1 & 0 & \dots\\
                    1 & 0 & 0 & 2\omega & \dots\\
                    0 & 0 & 0 & 1 & \dots\\
                    0 & 1 & -2\omega & 1 & \dots
                \end{pmatrix}
            \end{aligned}
        \]
        This already has full rank, so the system is controllable.

        \item Consider the eigenvectors of \(A^T\) and \(\ker((Bu)^T)\).
        
        \[
            Bu = \begin{pmatrix}
                0\\
                u_1\\
                0\\
                u_2
            \end{pmatrix}
        \]

        Excuse the notation.

        \[
            \ker(Bu) = \begin{pmatrix}
                x_1\\
                0\\
                x_3\\
                0
            \end{pmatrix}
        \]
        One of the eigenvectors of \(A^T\) is \(\begin{pmatrix}
            2\omega\\
            0\\
            -1\\
            1
        \end{pmatrix}\).

        As is \(\ker(Bu) \cap v = \{0\}\), so it is controllable.

        If \(u_1 = 0\), then \(\ker(Bu) = \begin{pmatrix}
            x_1\\
            x_2\\
            x_3\\
            0
        \end{pmatrix}\).

        In this case, \(\ker(Bu) \cap v = \{0\}\), so it is controllable.

        If \(u_2 = 0\), then \(\ker(Bu) = \begin{pmatrix}
            x_1\\
            0\\
            x_3\\
            x_4
        \end{pmatrix}\).

        In this case, \(\ker(Bu) \cap v \neq \{0\}\), so it is uncontrollable.
    \end{enumerate}
\end{solution}

\newpage

\begin{problem}{5}
    Consider an LTI system with realization
    \[
        A = \begin{pmatrix}
            -1 & 0\\
            0 & -1
        \end{pmatrix}, \quad
        B = \begin{pmatrix}
            -1\\
            1
        \end{pmatrix}, \quad
        C = \begin{pmatrix}
            1 & 0\\
            0 & 1
        \end{pmatrix}, \quad
        D = \begin{pmatrix}
            2\\
            1
        \end{pmatrix}
    \]
    Is this realization controllable? If not, perform a controllable decomposition to obtain a controllable realization of the same transfer function.
\end{problem}

\begin{solution}
    \[
        \tilde{C} = \begin{pmatrix}
            -1 & 1\\
            1 & -1
        \end{pmatrix}
    \]
    rank(\(\tilde{C}\)) = 1, so the system is uncontrollable.

    \(\tilde{V} = \begin{pmatrix}
        -1\\
        1
    \end{pmatrix}\).
    \[
        \begin{aligned}
            \dot{\tilde{x}} &= T^{-1}ATx + T^{-1}Bu\\
            y &= CT(T^{-1}x) + Du
        \end{aligned}
    \]
    So,
    \[
        \begin{aligned}
            \dot{\tilde{x}} &= \begin{pmatrix}
                A_c & A_{12}\\
                0 & A_u
            \end{pmatrix} \tilde{x} + \begin{pmatrix}
                B_c\\
                0
            \end{pmatrix} u\\
        \end{aligned}
    \]
    Find things:
    \[
        \begin{aligned}
            A\tilde{V} &= \begin{pmatrix}
                1\\
                -1
            \end{pmatrix} = \tilde{V} A_c \implies A_c = -1\\
            B &= \begin{pmatrix}
                -1\\
                1
            \end{pmatrix} = \tilde{V} B_c \implies B_c = 1\\
            CT &= T = \begin{pmatrix}
                C_c & C_u
            \end{pmatrix}
        \end{aligned}
    \]
    Note,
    \[
        T = \begin{pmatrix}
            -1 & 1\\
            1 & 1
        \end{pmatrix}
    \]
    So,
    \[
        C_c = \begin{pmatrix}
            -1\\
            1
        \end{pmatrix}
    \]
    Realiation:
    \[
        \begin{aligned}
            \begin{pmatrix} C_c & C_u \end{pmatrix} 
                (sI - \begin{pmatrix} A_c & A_{12} \\ 0 & A_u \end{pmatrix})^{-1} 
                \begin{pmatrix} B_c \\ 0 \end{pmatrix} + D 
            &=\begin{pmatrix} 
                C_c & C_u 
            \end{pmatrix} 
            \begin{pmatrix} 
                (sI - A_c)^{-1} & \tilde{A}_{21} \\ 
                0 & (sI - A_u)^{-1} 
            \end{pmatrix} 
            \begin{pmatrix} B_c \\ 0 \end{pmatrix} + D \\
            &= \begin{pmatrix} 
                C_c & C_u 
            \end{pmatrix} 
            \begin{pmatrix} 
                (sI - A_c)^{-1}B_c \\ 
                0 
            \end{pmatrix} + D \\
            &= C_c (s - A_c)^{-1} B_c + D\\
            &= \frac{1}{s + 1} \begin{pmatrix} -1 \\ 1 \end{pmatrix} + \begin{pmatrix} 2 \\ 1 \end{pmatrix}.
        \end{aligned}
    \]



\end{solution}

\end{document}