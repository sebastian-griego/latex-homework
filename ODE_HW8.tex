\documentclass[12pt]{article}
\usepackage[utf8]{inputenc}
\usepackage{amsmath, amssymb, amsthm}
\usepackage{enumitem}
\usepackage{geometry}
\usepackage{fancyhdr}
\usepackage{pgfplots}
\usepackage{tikz}
\usepackage{float}
\usepackage{graphicx}
\DeclareMathOperator{\Tr}{Tr}
\DeclareMathOperator{\rng}{rng}
\DeclareMathOperator{\norm}{||}
\DeclareMathOperator{\NN}{\mathbb{N}}
\DeclareMathOperator{\ZZ}{\mathbb{Z}}
\DeclareMathOperator{\QQ}{\mathbb{Q}}
\DeclareMathOperator{\RR}{\mathbb{R}}
\DeclareMathOperator{\CC}{\mathbb{C}}

% Page setup
\setlength{\headheight}{15pt}
\geometry{letterpaper, margin=1in}
\setlength{\parindent}{0pt}
\setlength{\parskip}{1em}
\pagestyle{fancy}
\fancyhf{}
\fancyhead[L]{\textbf{Sebastian Griego}}  % Replace with your name
\fancyhead[C]{\textbf{ODEs}}  % Replace with your course name
\fancyhead[R]{\textbf{Assignment \#8}}  % Replace with your assignment number
\fancyfoot[C]{\thepage}

\newenvironment{problem}[1]{
    \textbf{Problem #1:}
}{
    \rmfamily \vspace{1em}
}

\newenvironment{solution}{
    \textbf{Solution:}
    
}{
    
    \vspace{2em}
}

\begin{document}

\title{Homework \#8}  % Replace with the homework number
\author{Sebastian Griego}  % Replace with your name
\maketitle


\begin{problem}{1}
    Prove that when all the eigenvalues of \(A\) have strictly negative real parts,
    then there exists constants \(c, \lambda > 0\) such that
    \[
        \norm{e^{At}} \leq c e^{-\lambda t} \norm, \quad \forall t \in \RR.
    \]
\end{problem}

\begin{solution}
    Let \(\lambda_1, \ldots, \lambda_n\) be the eigenvalues of \(A\). Let
    \[
        A = VJV^{-1}, \quad J = \Lambda + N, \quad \Lambda N = N\Lambda, \quad N^p=0
    \]
    Note that \(N = 0\) if there is no Jordan form. The proof is the same/simpler.

    Then,
    \[
        e^{At} = V e^{\Lambda t}e^{Nt} V^{-1}.
    \]
    Define:
    \[
        \lambda_m = \min_j |Re(\lambda_j)|
    \]
    Then,
    \[
        \norm{e^{At}}\norm = \norm{V e^{\Lambda t}e^{Nt} V^{-1}}\norm \leq \norm{V}\norm \cdot \norm V^{-1}\norm \cdot e^{-\lambda_m t} \cdot \norm{e^{Nt}}\norm
    \]
    Where:
    \[
        \norm{e^{Nt}}\norm = \left\Vert \sum_{j=0}^{p-1} \frac{N^j t^j}{j!} \right\Vert \leq \sum_{j=0}^{p-1} \frac{t^j}{j!} \norm N \norm^j
    \]
    Note:
    \[
        \norm N \norm = 1 \implies \norm{e^{Nt}}\norm \leq \sum_{j=0}^{p-1} \frac{t^j}{j!}
    \]
    Back to the other thing:
    \[
            \norm{e^{At}}\norm \leq \norm{V}\norm \cdot \norm V^{-1}\norm \cdot e^{-\lambda_m t} \cdot \sum_{j=0}^{p-1} \frac{t^j}{j!} \\
    \]
    We can ignore the \(V\) and \(V^{-1}\) terms because they are constants.

    Define: \(0 < \epsilon < \lambda_m\). This gives:
    \[
        e^{-\lambda_m t} = e^{-(\lambda_m - \epsilon + \epsilon)t} = e^{-(\lambda_m - \epsilon)t} \cdot e^{-\epsilon t}
    \]
    So,
    \[
        \begin{aligned}
            e^{-\lambda_m t} \cdot \sum_{j=0}^{p-1} \frac{t^j}{j!} &= e^{-(\lambda_m - \epsilon)t} \cdot e^{-\epsilon t} \cdot \sum_{j=0}^{p-1} \frac{t^j}{j!} \\
            &= e^{-(\lambda_m - \epsilon)t} \left( \sum_{j=0}^{p-1} \frac{t^j}{j! \cdot e^{\epsilon t}} \right)
        \end{aligned}
    \]
    So, you can pick \(\lambda = \lambda_m - \epsilon\) and \(c\) that bounds the sum and \(V\) and \(V^{-1}\) terms such that:
    \[
        \norm{e^{At}}\norm \leq c e^{-\lambda t}, \quad \forall t \in \RR
    \]
\end{solution}

\newpage

\begin{problem}{2}
    Verify that \(\left(e^{At}\right)^T = e^{A^T t}\)
\end{problem}

\begin{solution}
    Using \((A^k)^T = (A^T)^k\) and \((A + B)^T = A^T + B^T\):
    \[
        \begin{aligned}
            \left(e^{At}\right)^T &= \left(\sum_{j=0}^{\infty} \frac{A^j t^j}{j!}\right)^T \\
            &= \sum_{j=0}^{\infty} \frac{(A^T)^j t^j}{j!} \\
            &= e^{A^T t}
        \end{aligned}
    \]
\end{solution}

\newpage

\begin{problem}{3}
    Consider the continuous-time LTI system
    \[
        \dot x = Ax, \quad x \in \RR^n
    \]
    and suppose there exists a positive constant \(\mu\) and positive-definite matrices \(P, Q \in \RR^n\) for which the Lyapunov equation
    \[
        A^T P + P A + 2\mu P = -Q
    \]
    holds. Show that all eigenvalues of \(A\) have real parts less than \(-\mu\).
\end{problem}

\begin{solution}
    Given this hint: Start by showing that all eigenvalues of \(A\) have real parts less than \(-\mu\) if and
    only if all eigenvalues of \(A + \mu I\) have real parts less than 0 (i.e., \(A + \mu I\) is a stability
    matrix).

    Proof of the hint:
    \[
        \begin{aligned}
            Av &= \lambda v \\
            Av + \mu v &= \lambda v + \mu v \\
            (A + \mu I)v &= (\lambda + \mu)v
        \end{aligned}
    \]
    \[
        Re(\lambda(A)) < -\mu \iff Re(\lambda(A) + \mu) < 0 \iff Re(\lambda(A + \mu I)) < 0
    \]

    Now the actual problem:

    Set up:
    \[
        \begin{aligned}
            A^T P + P A + 2\mu P &= -Q \\
            (A + \mu I)^T P + P (A + \mu I) &= -Q
        \end{aligned}
    \]
    Theorem: all evals of \(A + \mu I\) have real parts less than 0. The above hint then gives:
    \[
        Re(\lambda(A)) < -\mu \iff Re(\lambda(A + \mu I)) < 0
    \]

\end{solution}

\newpage

\begin{problem}{4}
    Consider the system
    \[
        \dot x = Ax + Bu, \quad y = Cx
    \]
    with
    \[
        A := \begin{pmatrix}
            0 & 1\\
            -1 & -2 \end{pmatrix}, \quad B := \begin{pmatrix}
                0\\
                1
            \end{pmatrix}
    \]
    \begin{enumerate}[label=(\alph*)]
        \item Given a \(1 \times 2\) matrix \(f:= \begin{pmatrix}
            f_1 & f_2
        \end{pmatrix}\), compute the characteristic polynomial of \(A + Bf\)
        \item Select \(f_1\) and \(f_2\) so that the eigenvalues of \(A + Bf\) are both zero.
        \item For the matrix \(f\) computed above, is the closed-loop system
        \[
            \dot x = (A + Bf)x
        \]
        stable?
    \end{enumerate}
\end{problem}

\begin{solution}
    \[
        \begin{aligned}
            A + Bf &= \begin{pmatrix}
                0 & 1\\
                -1 & -2
            \end{pmatrix} + \begin{pmatrix}
                0 & 0\\
                f_1 & f_2
            \end{pmatrix} \\
            &= \begin{pmatrix}
                0 & 1\\
                -1 + f_1 & -2 + f_2
            \end{pmatrix}
        \end{aligned}
    \]
    The characteristic polynomial is
    \[
        \begin{aligned}
            \lambda^2 - (\Tr(A + Bf)) \lambda + \det(A + Bf) &= 0 \\
            \lambda^2 - (-2 + f_1) \lambda + (1 - f_1) &= 0 \\
            \lambda^2 - (f_1 - 2) \lambda + (1 - f_2) &= 0\\
            \lambda &= \frac{1}{2} \left( (f_1 - 2) \pm \sqrt{(f_1 - 2)^2 - 4(1 - f_2)} \right)\\
            &= \frac{1}{2} \left( f_1 - 2 \pm \sqrt{(f_1^2 - 4f_1 + 4) + 4f_2 - 4} \right)\\
            &= \frac{1}{2} \left( f_1 - 2 \pm \sqrt{f_1^2 - 4f_1 + 4f_2} \right)
        \end{aligned}
    \]
    To get the eigenvalues to be zero, we need:
    \[
        \begin{aligned}
            f_1^2 - 4f_1 + 4f_2 &= 0 \\
            f_1 &= 2
        \end{aligned}
    \]
    Solve for \(f_2\):
    \[
        \begin{aligned}
            4 - 8 + 4f_2 &= 0 \\
            f_2 &= 1
        \end{aligned}
    \]
    So, \(f = \begin{pmatrix}
        2 & 1
    \end{pmatrix}\).

    The closed-loop system is:
    \[
        \begin{aligned}
            \dot x &= (A + Bf)x \\
            &= \begin{pmatrix}
                0 & 1\\
                -1 + f_1 & -2 + f_2
            \end{pmatrix}x\\
            &= \begin{pmatrix}
                0 & 1\\
                1 & -1
            \end{pmatrix}x
        \end{aligned}
    \]
    The trace and determinant are both negative, so the system is a saddle. So it is not stable.
\end{solution}


\end{document}
